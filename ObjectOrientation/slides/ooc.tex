\documentclass{beamer}

\usepackage[utf8x]{inputenc}
\usepackage{default}
\usetheme{PaloAlto}

\title{Orientação a objetos\\ \textbf{Orientação a objetos em ANSI C - 1}}
\author{Paulo Meirelles, Rodrigo Siqueira}
\date{\today}
\institute{\textbf{Universidade de Brasília - Faculdade do Gama}} 

\begin{document}

%SLIDE INICIAL DE APRESENTAÇÃO
\begin{frame}
  \titlepage
\end{frame}
  
%SLIDES == INTRODUÇÃO
\section{Introdução}
\begin{frame}
  \frametitle{Introdução}
  \begin{itemize}
   \item --- PENSAR ---.
  \end{itemize}
\end{frame}

%SLIDES == EXEMPLO
\section{Tipo de dados}
\begin{frame}
  \frametitle{O que é tipo de dados?}
  \begin{itemize}
    \item Pode ser visto como um conjunto de valores.
    \item Pode ser visto como um conjunto de valroes, mais operações para 
      trabalhar com eles.
  \end{itemize}
\end{frame}

\begin{frame}
 \frametitle{Boas práticas}
 \begin{block}{Primeira boa prática}
  Esconda a representação dos itens de dados e declare apenas os manipuladores 
  necessários.
 \end{block}
\end{frame}

% TIPOS DE DADOS ABSTRATO
\section{Tipo de dados abstrato}
\begin{frame}
 \frametitle{Dados abstrato}
 \begin{block}{Dados abstrato}
  Chamamos um tipo de dado abstrato, se nos não revelamrmos a sua representação 
  para o usuário.
 \end{block}
 \begin{block}{Dividir e conquistar}
  Esconda a informação e divida para conquistar.
 \end{block}
\end{frame}

% Exemplo
\section{Exemplo}
\begin{frame}
  \frametitle{--Explicar o exemplo usando conjuntos--}
\end{frame}

\begin{frame}
  \frametitle{Funções}
  --Explicar as funções
\end{frame}

% GERENCIAMENTO DA MEMÓRIA
\section{Gerenciamento da memória}
\begin{frame}
 \frametitle{Gerenciamento da memória}
 \begin{enumerate}
  \item Como obter um tipo \textit{Set}?
  \begin{itemize}
   \item R: Utilizando ponteiros para referir-se a conjuntos e elmentos.
  \end{itemize}
 \end{enumerate}

\end{frame}

\begin{frame}
 \frametitle{Funções}
  % Mostrar as funções e ir para o código
\end{frame}

\begin{frame}
 \frametitle{O que fazer com \textit{Set}?}
\end{frame}

% EXEMPLO DE UMA APLICAÇÃO
\begin{frame}
 \frametitle{Código da \textit{main}}
\end{frame}

\section{Implementando uma simplificação de new e delete}
\begin{frame}
 \frametitle{Como funicionará o \textit{new}}
  Se um objeto não armazena informação e se todo objeto pertence para pelo menos 
  um conjunto nos podemos repreentar cada objeto e cada conjunto como um valor 
  pequeno, único, inteiro, positivo utilizando como índice em um \textit{heap}. 
  Os objetos apontam para o conjunto que os contém.
\end{frame}



\end{document}

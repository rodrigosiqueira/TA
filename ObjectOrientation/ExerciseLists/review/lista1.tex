\documentclass[a4paper,10pt]{article}
\usepackage[utf8x]{inputenc}
\usepackage{graphicx}
\usepackage{pgf}
\pgfdeclareimage[height=1cm]{myimage}{p1_q1.png}


%opening   
\title{ \textbf{Universidade de Brasília -- Faculdade do Gama \\ Orientação a 
objetos \\ Revisão }}
\date{}

\begin{document}
\maketitle

\paragraph{}
\textbf{Atenção:} Esta lista tem por objetivo relembrar conceitos de ponteiro e 
aguçar a compreensão sobre o mesmo. Esta é uma lista opcional e \textbf{não} 
será atribuído qualquer ponto adicional por ela, sendo indicada para aqueles 
alunos que desejam aprimorar os seus conhecimentos. \\
Esta lista tem como meta:
\begin{itemize}
 \item Relembrar alguns conceitos de ponteiros.
\end{itemize}

\section{Exercícios de revisão}

\begin{enumerate}
 \item Faça um algoritmo que calcule e imprima o tamanho em \emph{bytes} de um 
  tipo de dados usando aritmética de ponteiros. Crie um ponteiro para cada um 
  dos tipos de dados a seguir: \emph{float}, \emph{double}, \emph{int}, 
  \emph{short}, \emph{long} e \emph{char}. Analise cuidadosamente os diferentes 
  resultados.\\
  \textbf{Dica:} Comece a verificar os tamanhos partindo do \emph{char} 
  e utilize o resultado obtido como base para os demais. Por fim recomenda-se 
  verificar o tamanho dos tipos ponteiros utilizando a macro sizeof (ex.:char 
  *,   float *, void *).

 \item Um IP (\emph{Internet Protocol}) nada mais é que uma identificação de um 
  dispositivo em uma rede local ou pública. Na sua versão 4, este é composto por 
  um endereços de até 32 bits. Faça um programa que tenha um vetor de char com 
  4 elementos, onde cada um representa uma parte do número IP. Com base em tal 
  vetor, insira cada um dos elementos dentro de uma única variável int (sabe-se 
  que um int tem 4 bytes em uma aquitetura 32 bits):\\
      Ex.: IP de uma máquina da FGA: 164.41.33.152\\
	  char fields[] = {152, 33, 41, 164};\\
	  Número inteiro que representa o IP:\\
	  Byte 0 = 152	// 0x98 em hexa\\
	  Byte 1 = 33	// 0x21\\
	  Byte 2 = 41 	// 0x29\\
	  Byte 3 = 164	// 0xA4\\
	  ou IP = 0xA4292198\\
      Observação: Preste atenção ao total de bytes de cada tipo primitivo da 
      sua máquina (o exercício anterior é útil neste sentido) e ao tipo de 
      organização da sua arquitetura (\emph{little ending ou big ending}).
      \textbf{Dica:} Utilize tipos sem sinal e inicialize os valores iniciais.

  \item É comum que no processo de codificação sejam criadas funções 
    para trabalhar com um determinado tipo nativo (por exemplo, \emph{int}), 
    contudo, conforme o código passa a crescer surge a necessidade que
    tal função também trabalhe com outros tipos nativos. Como alternativa, 
    muitos programadores replicam o trecho de código com o novo tipo ou buscam 
    alguma solução que exige a implementação de mais código. Neste contexto  
    crie uma programa que receba um vetor como parâmetro e um valor chave para 
    ser procurado dentro deste vetor. Faça com que esta função aceite qualquer 
    tipo de vetor, por exemplo, ela deve aceitar um vetor de \emph{int} ou 
    \emph{double} sem precisar ser alterada. Use no máximo cinco parâmetros e de 
    como retorno em caso de sucesso na busca o valor da posição no vetor ou -1 
    no caso de nada ser encontrado. (Ignore o caso de repetição das chaves, 
    isto é, retorne a primeira chave encontrada).\\
    \textbf{Dica:} Utilize o referências do tipo \textbf{void *}.

\end{enumerate}

\end{document}

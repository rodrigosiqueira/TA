\documentclass{beamer}

\usepackage[utf8x]{inputenc}
\usepackage{default}
\usetheme{PaloAlto}

\title{Sistemas Digitais 2\\ \textbf{Projeto de Circuitos Sequenciais}}
\author{Rodrigo Siqueira}
\date{Brasíla, 13/2011}
\institute{\textbf{Universidade de Brasília - Faculdade do Gama}} 

\begin{document}

%SLIDE INICIAL DE APRESENTAÇÃO
\begin{frame}
  \titlepage
\end{frame}
  
%SLIDES == INTRODUÇÃO
\section{Introdução}
\begin{frame}
  \frametitle{Introdução}
  \begin{itemize}
   \item Quando realizamos a análise de um circuito, queremos obter a descrição do mesmo. \pause
   \item Se quisermos construir o circuito, o que devemos fazer? \pause
   \item Para construir o circuito fazemos praticamente o contrário da análise.
  \end{itemize}
\end{frame}

\begin{frame}
 \frametitle{Introdução}
 O processo de projetar um circuito sequência pode ser dividido em sete etapas, são elas:
 \begin{block}{Etapas}
  \begin{enumerate}
   \item Partindo da descrição do circuito produza o diagrama de estados.\pause
   \item Derive a tabela de próximo estado do diagrama de estados.\pause
   \item Converta a tabela de próximo estado em uma tabela de implementação.\pause
   \item Derive as equações de excitação para cada entrada de flip-flop da tabela de implementação.\pause
   \item Derive a tabela de saída do diagrama de estados.\pause
   \item Derive as equações de saída da tabela de saída. \pause
   \item Desenhe o diagrama do circuito baseado nas equações de excitação e saída.
  \end{enumerate}

 \end{block}

\end{frame}


%SLIDES == EXEMPLO
\section{Exemplo}
\begin{frame}
  \frametitle{Exemplo}
  Desenvolvemos um exercício no qual aplicamos estes passos.\pause
  \begin{center}
   \begin{block}{Exemplo}
    Projete um circuito sequencial que conte a seguinte sequencia: \\3, 7, 2, 6, 3, 7, 2, 6 … \\ Teremos uma entrada C que inicializa ou termina o circuito. 
    Todos os estados não utilizados devem ser levados ao estado inicial, ou seja, o estado 3. Utilizar FF D e representar a contagem diretamente dos FF.
   \end{block}
  \end{center}

\end{frame}

%SLIDES == EXEMPLO DIAGRAMA DE ESTADOS
\section{Exemplo(Diagrama de estados)}
\begin{frame}
  \frametitle{Exemplo(Diagrama de estados)}
  \begin{itemize}
   \item Analisando o problema notamos que só temos 4 números diferentes (3, 7, 2, 6). Notamos que o maior elemento é o 7, que pode ser representado por 3 bits.\pause
  \end{itemize}

\end{frame}

\end{document}

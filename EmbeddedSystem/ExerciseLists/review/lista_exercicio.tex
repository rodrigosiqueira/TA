\documentclass[a4paper,10pt]{article}
\usepackage[utf8x]{inputenc}
\usepackage{graphicx}
\usepackage{pgf}
\pgfdeclareimage[height=1cm]{myimage}{p1_q1.png}


%opening   
\title{ \textbf{Universidade de Brasília -- Faculdade do Gama \\ Sistemas embarcados \\ Lista 1 }}
\date{Setembro - 2011}

\begin{document}
\maketitle

\section{Parte 1}
\begin{enumerate}
 \item Faça um algoritmo que calcule e imprima o tamanho em bytes de um tipo de dados usando aritmética de ponteiros. Crie um ponteiro para cada 
      um dos tipos: float, double, int, short, long e char. Analise cuidadosamente os diferentes resultados.
 \item Um IP (Internet Protocol) nada mais é que uma identificação de um dispositivo em uma rede local ou pública. Na sua versão 4, este é composto 
      por um endereços de até 32 bits. Faça um algoritmo que, utilizando ponteiros, crie um número IP a partir de 4 bytes lidos de um array de char. 
      Um número IP é um inteiro,onde cada byte que compõe o inteiro armazena um dos campos do IP (usar apenas ponteiros).\\
      Ex.: IP de uma máquina da FGA: 164.41.33.152\\
	  char fields[] = {152, 33, 41, 164};\\
	  Número inteiro que representa o IP:\\
	  Byte 0 = 152 // 0x98 em hexa\\
	  Byte 1 = 33 // 0x21\\
	  Byte 2 = 41 // 0x29\\
	  Byte 3 = 164 // 0xA4\\
	  ou IP = 0xA4292198\\
      Observação: Preste atenção ao total de bytes de cada tipo primitivo da sua máquina (o exercício anterior é útil neste sentido) e ao tipo de 
      organização da sua arquitetura (little ending ou big ending).
 \item O formato de imagem PGM (Portable GreyMap) é um padrão bitmap que contém somente quatro linhas de cabeçalho, dados armazenados como unsigned 
      char fornecendo no máximo uma escala de cinza de 256 níveis. A estrutura geral do PGM é constituída de:\\
      \begin{itemize}
       \item Linha 1: Contém a assinatura do arquivo da imagem e identificador do arquivo PGM.
       \item Linha 2: É uma linha de comentário.
       \item Linha 3: Fornece informações sobre o número de linhas e colunas de dados armazenados em um arquivo.
       \item Linha 4: Especifica o nível máximo de cinza contido na imagem.
      \end{itemize}
      Um exemplo da estrutura de uma imagem no formato PGM:\\
	\\P5 \\
	\#FGA\/SEM imagem gerada Ver.1.0 (0) \\
	10 10 \\
	255 \\
	0 1 2 3 4 5 6 7 8 9 10 11 12 13 14 15 16 17 18 19 20 21 22 23 24 25 \\
	26 27 28 28 30 31 32 33 34 35 36 37 38 39 40 41 42 43 44 45 46 47 \\
	48 49 50 51 52 53 54 55 56 57 58 59 60 61 62 63 64 65 66 67 68 69 70 \\ 
	71 72 73 74 75 76 77 78 79 80 81 82 83 84 85 86 87 88 89 90 91 92 93 \\
	255 255 255 255 255 255 \\
      Um efeito muito comum que costuma-se dar a esta imagem é o chamado negativo. Este faz com que a cor de cada pixel da imagem original se 
      transforme na cor inversa (por exemplo, um pixel branco se transforma em preto), a cor inversa é o valor da subtração entre 255 e o valor 
      do pixel. Com base nas informações dadas, faça um programa que leia uma imagem no formato PGM, tire o negativo dela e salve-a como uma nova
      imagem.
 \item $<<$Questão explicada de como criar uma lista encadeada$>>$
\end{enumerate}

\section{Parte 2}
  \begin{enumerate}
   \item $<<$Enunciado do Problema de produtor e consumidor$>>$
   \item $<<$Problema do jantar dos filósofos$>>$
   \item $<<$Problema dos leitores e escritores$>>$
   \item $<<$Problema do barbeiro sonolento$>>$
  \end{enumerate}


\section{Parte 3}
  \begin{enumerate}

   \item O \emph{Hypertext Transfer Protocol} (HTTP) - Protocolo de Transferência de Hipertexto - é um protocolo de aplicação responsável pelo tratamento 
	de pedidos e respostas entre cliente e servidor na \emph{World Wide Web}. Veja abaixo um pequeno exemplo da sintaxe básica para trabalhar com 
	tais páginas:\\
\begin{verbatim}
 	<HTML>
	  <HEAD>
	    <TITLE>$Pagina de teste$</TITLE>
	  </HEAD>
          
	  <BODY>
	    <H2>PID desse Web Server: 12  <H2>
	    <H5>Hora da requisicao: 12:27 </H5>
	  </BODY>
	</HTML>
\end{verbatim}
	Existem programas de computador responsáveis por aceitar pedidos HTTP de clientes, geralmente os navegadores, e servi-los com respostas HTTP,
	incluindo opcionalmente dados que geralmente são páginas web, tais como documentos HTML com objetos embutidos (imagens, etc.). Tais programas 
	são chamados de servidor web e são responsável por aceitar tais pedidos HTTP de clientes. Crie um pequeno servidor web que leia uma arquivo 
	\emph{index.html}(contendo o código acima ou similar) e permita que outros usuários visualizem no seu navegador tal página. \\
	\textbf{\emph{Dicas:}}
	\begin{itemize}
	 \item Recomenda-se a utilização da porta 3333, contudo não há fortes restrições quanta a esta.
	 \item Realize uma pesquisa sobre servidores web, como exemplo busque pelo servidor Apache.
	 \item Trabalhe inicialmente com \emph{tags} HTML simples.
	\end{itemize}

   \item Um \emph{chat} pode ser definido de forma geral como uma ferramenta de comunicação, disponível online, utilizada para realizar comunicação
	em tempo real. Normalmente um \emph{chat} que promove a interação entre vários clientes costuma ter um servidor que gerencia os pedidos de 
	conexão e repassa as diversas mensagens aos clientes. Realize as seguintes tarefas:
	\begin{itemize}
	 \item Crie um programa servidor que aceite e mantenha a conexão de pelo menos quatro clientes (\emph{Dica: Para cada pedido de conexão crie 
	    um processo novo.}).
	 \item Faça com que cliente e servidor troquem mensagens. 
	 \item Faça com que os clientes troquem mensagens entre si por meio do servidor. 
	\end{itemize}
   \item Um \emph{port scanner} (scanner de porta) é um aplicativo com o objetivo de testar as portas lógicas de determinado \emph{host} remoto. 
	Neste teste ele identifica o status das portas, se estão fechadas, escutando ou abertas. Pode-se explicitar o range de portas que o 
	aplicativo irá scanear, por ex: 25 a 80. Geralmente os \emph{port scanners} são usado por pessoas mal intencionadas para identificar 
	portas abertas e planejar invasões. Pode também ser usado também por empresas de segurança para análise de vulnerabilidades(\emph{pen test}).
	Faça um \emph{port scanner} simples que trabalhe somente com os serviços TCP.\\
	\textbf{Observação: Recomenda-se que tal programa seja testado na sua própria rede local, valendo observar que qualquer código produzido 
	é de inteira responsabilidade do criador. O exercício proposto só deve ser utilizado para fins de estudo.}
  \end{enumerate}
\section{Parte 4 - Adicional}
\begin{enumerate}
  \item O diretor de um banco resolver modificar a forma de atender os clientes, considerando idade e sexo de cada um. As regras estabelecidas 
	para acesso ao banco e pagamento de contas foram as seguintes:
      \begin{itemize}
       \item Não pode haver mais do que duas mulheres (de qualquer idade) pagando contas ao mesmo tempo no banco.
       \item Crianças só podem realizar seus pagamentos depois que algum adulto já o fez.
       \item Quando um homem idoso (idade igual ou superior a 60 anos) estiver pagando contas, a prioridade é específica para ele (não pode haver 
	    mulher ou crianças utilizando os caixas junto com os idosos). Portanto, se houver alguma mulher ou criança realizando pagamentos, o 
	    idoso deve aguardar até que esses sejam atendidos para depois entrar e realizar seus pagamentos.
       \item Até três crianças podem entrar juntas no banco para pagarem suas contas.
       \item Não mais do que quatro homens idosos podem entrar simultaneamente no banco para pagarem suas contas.
       \item Toda combinação de clientes que não burlar alguma das regras acima é valida.
      \end{itemize}
	Crie um programa que simule as regras de acesso ao banco, considerando os seguintes requisitos:
      \begin{enumerate}
       \item Usar memória compartilhada e semáforos em ambiente Linux/Linguagem C para a gestão do acesso ao banco. Obs.: Considere a criação de 
	    um único identificador de semáforo contemplando vários semáforos, se for o caso.
       \item O número de clientes do banco (homens, mulheres, crianças) com suas respectivas idades pode ser informado via teclado ou ser gerado 
	    randomicamente.
       \item O programa deve gerar tantos processos filhos quantos cliente houverem para esse banco (cada processo pode representar um idoso, uma 
	    mulher, uma criança e assim por diante).
       \item Antes de tentar entrar no banco, o cliente deve pensar um pouco sobre as contas que vai pagar durante um tempo aleatório (entre 0 e 
	    1000 microsegundos).
       \item O procedimento de uso dos serviços do banco (pagamento, etc.) pode ser simulado por uma instrução "usleep()" de 50 micro-segundos em 
	    conjunto com algum texto impresso na tela.
       \item O processo pai deve aguardar o término de todos os processos filhos. Quando estes terminarem, o processo pai deve remover o segmento 
	    de memória compartilhado e os semáforos criados.
       \item Coloque instruções de saída em tela nos processos para crianças, mulheres, adultos e idosos de forma a acompanhar em que etapa cada 
	    um deles está. Execute o programa gerado várias vezes. Para cada execução varie o número de clientes. Crie um quadro adequado para 
	    apresentar os resultado a serem visualizados. 
      \end{enumerate}
\end{enumerate}


\end{document}

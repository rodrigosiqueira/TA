\documentclass[a4paper,10pt]{article}
\usepackage[utf8x]{inputenc}
\usepackage{graphicx}
\usepackage{pgf}
\pgfdeclareimage[height=1cm]{myimage}{p1_q1.png}
\usepackage{listings}

%----------------- CONFIGURAÇÃO PARA OS COMANDOS -----------------------
\lstset
{
  basicstyle = \footnotesize, 		% Tamanho da fonte do código
  %numbers = left, 			% Posição da numeração das linhas
  %numberstyle = \tiny\color{blue}, 	% Estilo da numeração de linhas
  %stepnumber = 1, 			% Numeração das linhas ocorre a cada quantas linhas?
  %numbersep = 10pt, 			% Distância entre a numeração das linhas e o código
  backgroundcolor = \color{white}, 	% Cor de fundo
  showspaces = false, 			% Exibe espaços com um sublinhado
  showstringspaces = false, 		% Sublinha espaços em Strings
  showtabs = false, 			% Exibe tabulação com um sublinhado
  frame = single, 			% Envolve o código com uma moldura, pode ser single ou trBL
  rulecolor = \color{black}, 		% Cor da moldura
  tabsize = 2, 				% Configura tabulação em x espaços
  captionpos = b, 			% Posição do título pode ser t (top) ou b (bottom)
  breaklines = true, 			% Configura quebra de linha automática
  breakatwhitespace= false, 		% Configura quebra de linha
  title = \lstname, 			% Exibe o nome do arquivo incluido
  %caption = \lstname, % Também é possível usar caption no lugar de title
  keywordstyle = \color{blue}, 		% Estilo das palavras chaves
  commentstyle = \color{dkgreen}, 	% Estilo dos Comentários
  stringstyle = \color{mauve}, 		% Estilo de Strings
  escapeinside = {\%*}{*)}, 		% Permite adicionar comandos LaTeX dentro do seu código
  morekeywords     ={*,...} 		% Se quiser adicionar mais palavras-chave
}

%------------------------------------------------------------------------
%------------------------- CAPA -----------------------------------------
%------------------------------------------------------------------------
    \title{Como instalar o Linux no PenDrive com o Xenomai}
    \author{Rodrigo Siqueira de Melo, Evandro Leonardo Silva Teixeira}
    \date{27/12/2013}

%------------------------------------------------------------------------
%------------------------- ÍNDICE E DEMAIS ITENS ------------------------
%------------------------------------------------------------------------
    \begin{document}
    \maketitle
    \titlepage
    \tableofcontents

%------------------------------------------------------------------------
%------------- INFORMATIVOS SOBRE O TUTORIAL ----------------------------
%------------------------------------------------------------------------
\newpage
\section{Informativos sobre este tutorial}
  \subsection{Sobre este tutorial}
    \paragraph{}
    Seja bem vindo ao tutorial de instalação do \emph{Linux} com \emph{Xenomai} em em um pendrive. Antes de 
    começar a explicação sobre como proceder, gostaríamos de fazer algumas observações sobre este trabalho. 
    Seguem:
      \begin{itemize}
	\item \textbf{Nosso objetivo principal}: Este tutorial visa ensinar o leitor como instalar o 
	      \emph{Linux} em um pendrive e em seguida a configurar o \emph{Xenomai} no mesmo. Contudo o 
	      diferencial deste trabalho consiste em tentar ensinar o processo e não somente mostrar os 
	      passos. 
	\item \textbf{Estrutura deste documento}: Visando atender os objetivo deste material, dividimos este 
	      o trabalho em duas partes. Na primeira parte são mostrados os passos básicos com um nível de 
	      detalhamento intermediário, visando uma rápida execução. Na segunda parte, temos uma série de 
	      conceitos teóricos que fundamentam os passos explicados na primeira parte. Vale observar que as 
	      duas partes são interligadas, quando você tiver alguma dúvida ou passar por algum erro na primeira 
	      parte recomendamos fortemente que você faça uma leitura da segunda parte. O ideal serial uma 
	      leitura completa deste tutorial, contudo nada impede o leitor de executar os passos e realizar 
	      uma leitura seletiva sobre o que mais o interessa.
	\item \textbf{Sobre as referências}: Gostaríamos deixar claro que este tutorial só foi possível graças 
	      a enumeras referências coletadas pela \emph{Internet} e livros. Ao final deste documento citamos 
	      todas as referências bibliográficas utilizadas para criação deste material e os nomes daqueles que 
	      contribuirão diretamente para escrita deste documento. \\
	      \emph{Se eu vi mais longe, foi por estar de pé sobre ombros de gigantes. Isaac Newton}
      \end{itemize}
    \subsection{O que é preciso para executar este tutorial e onde foi testado}
      Para executar este tutorial alguns pré requisitos mínimos são necessários, são eles:
      \begin{itemize}
       \item Tutorial completo (Com \emph{Xenomai}): Pendrive de 2Gb, Computador com \emph{Ubuntu} ou \emph{Debian} 
	    instalado.
       \item Tutorial incompleto (só instalação do \emph{Linux}): Pendrive de 1Gb (existe algumas restrições quanto a
	    interface que será utilizada), Computador com \emph{Ubuntu} ou \emph{Debian} instalado.
       \item Ambos: Conexão com Internet e paciência.
      \end{itemize}
      Segue as condições na qual este tutorial foi testado:\\
      \begin{center}
	\begin{tabular}{|p{2cm}|p{2cm}|p{2cm}|p{2cm}|p{2cm}|}
	  \hline
	    Distribuição utilizada & Distribuição instalada no Pendrive & Tamanho do Pendrive & Interface & Arquitetura\\	
	  \hline
	    Debian & Squeeze, Lenny & 16Gb, 6Gb, 4Gb e 1Gb & OpenBox, Gnome, LXDE/OpenBox/Gnome e OpenBox & x86\\
	  \hline 
	    Ubuntu & Squeeze, Lenny & 16Gb, 2Gb & OpenBox, LXDE & x86\\
	  \hline
	\end{tabular}
       \end{center}

\newpage
%------------------------------------------------------------------------
%------------------------- SEÇÃO: PARTE 1 -------------------------------
%------------------------------------------------------------------------
\section{Parte 1 - Instalando e configurando o Linux no Pendrive}
%-------------- INSTAÇÃO DE PACOTE --------------------------------------
  \subsection{Instalação de pacotes básicos}
    \paragraph{}
    Alguns pacotes e programas são fundamentais para proceder com os passos iniciais. Portanto começamos 
    com a instalação dos seguintes \emph{softwares}:
      \begin{lstlisting}[frame=single]
	# apt-get install mbr
	# apt-get install syslinux
	# apt-get install debootstrap
	# apt-get install qemu
      \end{lstlisting}
    Vale observar que ao longo do tutorial será preciso instalar outros pacotes, contudo a maioria deles 
    ocorrerão em um contexto de \emph{enjaulamento}\footnote{Este conceito será explicado no decorrer das 
    atividades}. 

%-------------- LISTA DE PASSOS PARA PREPARAR O PENDRIVE ----------------
  \subsection{Preparando o pendrive}
    \begin{enumerate}
      \item Conecte o seu pendrive na porta USB.
      \item Para saber como o \emph{Linux} reconheceu o \emph{Pendrive} digite:
	\begin{lstlisting}[frame=single]
	  $ dmesg | tail
	\end{lstlisting}
      Você verá um resultado semelhante ao apresentado abaixo:

	\begin{lstlisting}
      [  268.014375] sd 5:0:0:0: Attached scsi generic sg2 type 0
      [  268.015556] sd 5:0:0:0: [sdb] 3942399 512-byte logical blocks: (2.01 GB/1.87 GiB)
      [  268.016267] sd 5:0:0:0: [sdb] Write Protect is off
      [  268.016274] sd 5:0:0:0: [sdb] Mode Sense: 45 00 00 08
      [  268.016280] sd 5:0:0:0: [sdb] Assuming drive cache: write through
      [  268.018169] sd 5:0:0:0: [sdb] Assuming drive cache: write through
      [  268.018177]  sdb: sdb1
      [  268.024556] sd 5:0:0:0: [sdb] Assuming drive cache: write through
      [  268.024566] sd 5:0:0:0: [sdb] Attached SCSI removable disk
      [  268.319097] FAT: utf8 is not a recommended IO charset for FAT filesystems, filesystem will be case sensitive!
	\end{lstlisting}
      Repare que esta saída está nos dizendo que o pendrive na máquina atual é reconhecida como \emph{sdb}. 
      Logo podemos realizar operações sobre ele pelo caminho \emph{/dev/sdb}.

    %-------------- ZERANDO O PENDRIVE --------------------------------------
      \subsubsection{Zerando o \emph{pendrive}}
	\item Para garantir a integridade, escrevemos 0 em todo o \emph{pendrive} por meio do comando abaixo:
	    \begin{lstlisting}[frame=single]
	      # dd if=/dev/zero of=/dev/xxx
	    \end{lstlisting}
	    Substitua o xxx pelo nome no qual o seu dispositivo foi reconhecido pelo sistema, por exemplo 
	    \emph{sdb}. Dependendo do tamanho do \emph{pendrive}, este passo pode levar um tempo e ao final 
	      apresenta uma saída semelhante a apresentada abaixo:
	    \begin{verbatim}
	      dd: escrevendo em "/dev/sdb": Não há espaço disponível no dispositivo
	      1953664+0 registros de entrada
	      1953663+0 registros de saída
	      1000275456 bytes (1,0 GB) copiados, 326,176 s, 3,1 MB/s
	    \end{verbatim}

    %-------------- CRIANDO PARTIÇÕES NO PENDRIVE/BOOTÁVEL ------------------
      \subsubsection{Criando partições no \emph{pendrive} e tornando-o bootável}
      Neste passo você pode utilizar qualquer ferramenta que você esteja habituado a usar para realizar 
      particionamento de disco. Contudo apresenta-se os passos com \emph{fdisk}
	\item Utilize o comando \emph{fdisk} para particionar o seu dispositivo:
	  \begin{lstlisting}
	    # fdisk /dev/xxx
	  \end{lstlisting}
	  Substitua \emph{/dev/xxx} pelo seu dispositivo.
  
    \item Quando estiver no \emph{prompt} de comando do \emph{fdisk} siga a sequencia abaixo:
      \begin{itemize}
	\item Digite \emph{\textbf{n}} e pressione \emph{\textbf{ENTER}}.
	\item Digite \emph{\textbf{p}} seguido da tecla \emph{\textbf{ENTER}}.
	\item Em número de partição, digite \emph{\textbf{1}} e depois \emph{\textbf{ENTER}}.
	\item Em primeiro cilindro digite \emph{\textbf{ENTER}}.
	\item Quando for pedido o número do último cilindro digite \emph{\textbf{ENTER}}.
	\item Você será levado de volta ao \emph{prompt} do \emph{fdisk}.
	\item Digite \emph{\textbf{a}} seguido de 1 e depois \emph{\bf{ENTER}} para tornar a 
	    partição \emph{bootável}. Em seguida digite \emph{\textbf{1}} e \emph{\textbf{ENTER}}
      \end{itemize}

    \item Quando estiver dentro do \emph{prompt} do \emph{fdisk} digite \emph{\textbf{p}} e \emph{\textbf{ENTER}} 
      para visualizar a partição criada. Você verá uma mensagem parecida com a apresentada abaixo:
	\begin{lstlisting}
	Disk /dev/sdb: 15.5 GB, 15504900096 bytes
	64 heads, 32 sectors/track, 14786 cylinders
	Units = cilindros of 2048 * 512 = 1048576 bytes
	Sector size (logical/physical): 512 bytes / 512 bytes
	I/O size (minimum/optimal): 512 bytes / 512 bytes
	Disk identifier: 0x78d8c044

	Dispositivo Boot Start   End   Blocks Id System
	/dev/sdb1   *      1   14786 15140848 83 Linux
	\end{lstlisting}

    \item Digite \emph{\textbf{w}} para salvar as alterações e sair.

    \item Por fim, digite:
      \begin{lstlisting}
      # fdisk /dev/xxx -l
      \end{lstlisting}
      Este comando mostrará como está o disco atual e permitirá a conferência do trabalho realizado.

    %-------------- CRIANDO UM MBR E FORMATANDO A PARTIÇÃO ------------------
      \subsubsection{Criando um MBR(\emph{Master Boot Record}) e formatando a partição}

	\item Digite o comando abaixo para instalar o MBR no seu \emph{pendrive} (Ele não apresenta 
	  saída no terminal):
	  \begin{lstlisting}
	  # install-mbr /dev/xxx
	  \end{lstlisting}

	\item Em seguida vamos formatar a partição para \emph{ext2}. Digite o comando abaixo:
	  \begin{lstlisting}
	  # mkfs.ext2 /dev/xxx1 -L RAIZ
	  \end{lstlisting}
	  Observe que definimos o rótulo RAIZ para a partição, este passo é fundamental para o que o
	  \emph{boot} ocorra de maneira correta.
    %-------------- CRIANDO UM SISTEMA DE BOOT ------------------------------
      \subsubsection{Criando um sistema de boot}  
	
	\item Monte a partição em \emph{mnt}:	
	  \begin{lstlisting}
	  # mount /dev/xxx1 /mnt
	  \end{lstlisting}

	\item Execute os comandos abaixo:
	  \begin{lstlisting}
	    # mkdir -p /mnt/boot/extlinux
	    # extlinux -i /mnt/boot/extlinux
	  \end{lstlisting}
      \end{enumerate}

%------------------------------------------------------------------------ 
%-------------- INSTALANDO O DEBIAN NO PENDRIVE -------------------------
%------------------------------------------------------------------------
  \subsection{Instalando o \emph{debian} no \emph{Pendrive}}
    \begin{enumerate}
%-------------- INSTALANDO O DEBIAN -------------------------------------
      \subsubsection{Instalando o debian}
      \item Para instalar o \emph{debian}, escolha uma versão e utilize um 
	    dos comandos abaixo para isto:\\
	Para o \emph{Squeeze}:
	\begin{lstlisting}
	  debootstrap --arch i386 squeeze mnt/ http://ftp.us.debian.org/debian
	\end{lstlisting}
	Para o \emph{Lenny}:
	\begin{lstlisting}
	  debootstrap --arch i386 lenny mnt/ http://archive.debian.org/debian/
	\end{lstlisting}
	Para o \emph{Etch}:
	\begin{lstlisting}
	  debootstrap --arch i386 etch mnt/ http://archive.debian.org/debian/
	\end{lstlisting}
	  \begin{center}
	    \begin{tabular}{|p{7cm}|}
	      \hline
		\textbf{ATENÇÃO}: Para este tutorial iremos utilizar o \emph{squeeze}. Vale observar que 
		alguns passos podem variar de distribuição para distribuição, principalmente aqueles
		que referem-se a instalação de pacotes. \\
	      \hline
	    \end{tabular}
	  \end{center}
%-------------- CONFIGURANDO O DEBIAN -----------------------------------
    \subsubsection{Configurando o debian}
    \item Ao término da instalação faremos o enjaulamento no diretório \emph{mnt}, isto fará com que 
	toda a ação executada após o comando ocorrera dentro do \emph{pendrive}. Para isto utilize o 
	comando abaixo:
	  \begin{lstlisting}[frame=single]
	  # chroot /mnt
	  \end{lstlisting}
	  \begin{center}
	    \begin{tabular}{|p{7cm}|}
	      \hline
		\textbf{ATENÇÃO}: Se o comando \emph{chroot} apresentar algum erro pode ser necessário um 
		passo adicional para utiliza-lo. Para isto execute o comando \emph{$echo$ 1 $> vdso\_enable$} 
		e altere o valor de 0 para 1.\\
	      \hline
	    \end{tabular}
	  \end{center}

    \item Monte o diretório \emph{proc} dentro do enjaulamento:
      \begin{lstlisting}
	# mount -t proc proc /proc
      \end{lstlisting}

%-------------- AJUSTE A LISTA DE REPOSITÓRIO ---------------------------
    \subsubsection{Ajuste a lista de repositório}
    \item digite:
      \begin{lstlisting}
	nano /etc/apt/source.list
      \end{lstlisting}
      Adicione as seguintes fontes:
	\begin{lstlisting}
	  deb http://ftp.us.debian.org/debian squeeze main contrib non-free

	  # squeeze-updates, previously known as 'volatile'
	  deb http://ftp.br.debian.org/debian/ squeeze-updates main
	  deb-src http://ftp.br.debian.org/debian/ squeeze-updates main
	\end{lstlisting}

    \item Após a alteração no arquivo digite:
      \begin{lstlisting}
	# apt-get update
      \end{lstlisting}

%-------------- DEFININDO A LÍNGA LOCAL ---------------------------------
    \subsubsection{Definindo a língua local}
    \item Digite os comandos abaixo para configurar questões de idioma:
      \begin{lstlisting}
	# apt-get install locales
	# dpkg-reconfigure locales
      \end{lstlisting}
    \item Na tela que será apresentada selecione: $pt\_BR ISO-8859-1$ e $pt\_BR.UTF-8 UTF-8$. Na tela 
	seguinte deixe como \emph{default} $pt\_BR.UTF-8$. 

%-------------- INSTALANDO UM KERNEL ------------------------------------
    \subsubsection{Instalando um kernel com \emph{Xenomai}}
    \item \textbf{\emph{Atenção:}}: Se você não tem interesse em instalar o \emph{Xenomai} pule para a próxima seção. 
      

    \subsubsection{Instalando um kernel (Sem a o \emph{Xenomai})}
    \item \textbf{\emph{Atenção}}: Se você fez a instalação do \emph{Kernel} com \emph{Xenomai}, pule esta seção.
    \item Verificar os \emph{kerneis} disponíveis para instalação:
      \begin{lstlisting}
	# apt-cache search ^linux-image
      \end{lstlisting}
      Veja a resposta do comando a cima e procure por um \emph{kernel} que se encaixe com as 
      suas configurações de hardware. Por exemplo, linux-image-2.6.26-1-686.
    \item Após selecionar um \emph{kernel}, digite:
      \begin{lstlisting}
	# apt-get install firmware-linux-nonfree
	# apt-get install linux-image-2.6.26-1-686
      \end{lstlisting}

	 \begin{center}
	    \begin{tabular}{|p{8cm}|}
	      \hline
		\textbf{ATENÇÃO}: Caso a seguinte pergunta aparece: \emph{Abort initrd kernel 
		    image installation?},responda Não. Se por engano você responder Sim, 
		    execute o comando.\\
	      \hline
		\begin{verbatim}
		  # update-initramfs -c -k [VersaoKernel]
		\end{verbatim}\\
	      \hline
	    \end{tabular}
	  \end{center}

%-------------- CRINAÇÃO DE UM ARQUIVO DE SWAP --------------------------
    \subsubsection{Criação de um arquivo de \emph{swap}}
    \item Vamos criar um \emph{swap} de 100Mb, digite o comando abaixo:
      \begin{lstlisting}
	# dd if=/dev/zero of=/swapfile bs=10M count=10
	# mkswap /swapfile
      \end{lstlisting}

%-------------- INTEGRAÇÃO DO KERNEL COM EXTLINUX E RAIZ ----------------
    \subsubsection{Integração do \emph{Sistema} com o \emph{bootloader} e a partição raiz}
    \item Crie o arquivo de configuração:
      \begin{lstlisting}
	# touch /boot/extlinux/extlinux.conf
      \end{lstlisting}
    
    \item Edite o arquivo recém criado com as configurações abaixo:
      \begin{lstlisting}
	DEFAULT linux
	LABEL linux
	SAY Inicializando o Debian Squeeze...
	KERNEL /vmlinuz
	APPEND ro root=LABEL=RAIZ initrd=/initrd.img
      \end{lstlisting}
      Repare que neste ponto passamos a ter um arquivo de configuração que referência o 
      dispositivo pelo rótulo RAIZ. Isto fará com que o \emph{pendrive} de o \emph{boot}
      corretamente.

    \item Editando o arquivo \emph{/etc/fstab}:
      \begin{lstlisting}
# nano /etc/fstab

#<file system> <mount point> <type> <options>    <dump> <pass>
proc            /proc         proc   defaults         0      0
LABEL=RAIZ      /             ext2   defaults,noatime 0      1
/swapfile       none          swap   sw               0      0
      \end{lstlisting}
%---- ADICIONANDO UM USUÁRIO, ALTERANDO A SENHA ROOT E TESTE ------------    
    \subsubsection{Adicionando um usuário, alterando senha de root e testando}
    \item Adicionando um novo usuário, preencha os campos solicitados pelo comando:
      \begin{lstlisting}
	# adduser NOME
      \end{lstlisting}

    \item Em seguida altere a senha de \emph{root}:
      \begin{lstlisting}
	# passwd root
      \end{lstlisting}

    \item Desmonte a pasta \emph{/proc} e saía do enjaulamento:
      \begin{lstlisting}
	# apt-get clean
	# umount /proc
	# exit
      \end{lstlisting}
    
    \item Desmonte o \emph{pendrive}
      \begin{lstlisting}
	# umount /mnt/
      \end{lstlisting}
    
     \item Neste ponto já temos um sistema operacional \emph{bootável}. Testando o \emph{pendrive}, 
	existem duas possibilidades:
      \begin{lstlisting}
	Op1: Reinicie o pendrive e de boot pelo computador.
	Op2: qemu -hda /dev/xxx
      \end{lstlisting}
  \end{enumerate}

%------------------------------------------------------------------------ 
%----------------------- AJUSTES BÁSICOS --------------------------------
%------------------------------------------------------------------------
  \subsection{Ajuste básicos}
    \begin{enumerate}
      \item Após o teste repetimos o enjaulamento:
	\begin{lstlisting}
	  # umount /dev/xxx
	  # mount /dev/xxx /mnt
	  # chroot /mnt
	  # mount -t proc proc /proc
	\end{lstlisting}

      \item (OPCIONAL)Caso queira alterar o nome local, digite o comando abaixo e insira o 
	  nome desejado no arquivo.
	\begin{lstlisting}
	  # nano /etc/hostname
	\end{lstlisting}

      \item (OPCIONAL)Altere o arquivo \emph{/etc/hosts}. Recomenda-se que seja realizado a cópia de 
	  algum computador que já tenha este arquivo. Do contrário configure da maneira que for 
	   conveniente para o seu contexto.
	\begin{lstlisting}
	  # nano /etc/hosts
	\end{lstlisting}

      \item Instalação de alguns pacotes mínimos. O ideal é que todos os pacotes sejam instalado, contudo 
	aqueles que possuem OP na frente são optativos e não influenciaram na instalação e podem ser evitados
	para \emph{pendrives} pequenos.
	\begin{lstlisting}
	# apt-get install build-essential
	# apt-get install ntpdate
	OP1 # apt-get install dosfstools ntfsprogs ntfs-3g jfsutils reiserfsprogs xfsprogs mbr testdisk
	OP2 # apt-get install bittwist dnsutils fping hping3 ipcalc netcat nmap tcpdump telnet traceroute wireless-tools
	# apt-get install ftp rsync ssh udpcast
	OP3 # apt-get install clamav console-tools ddrescue debootstrap durep fdupes grub hexedit \
	    hwinfo less lshw mc myrescue magicrescue parted rcconf strace unzip wipe zip
	\end{lstlisting}

      \item (OPCIONAL)Configure o arquivo \emph{/etc/network/interface} para obter uma rede assim o boot for realizado.
	Recomenda-se que seja feita uma cópia do arquivo do sistema operacional em uso, ou seja, o que 	
	está fora do enjaulamento do /emph{chroot}. 

      \item (OPCIONAL)Configuração básica da rede
	\begin{lstlisting}
	  # nano /etc/rc.local
	    Digite: dhclient
	\end{lstlisting}

      \item (OPCIONAL)Configurando um \emph{DNS} para resolver a Internet:
	\begin{lstlisting}
	  # apt-get install bind9
	  # nano /etc/resolv.conf 
	    Digite: nameserver 127.0.0.1
	  # nano /etc/dhcp3/dhclient.conf
	    Descomente: prepend domain-name-servers 127.0.0.1;
	\end{lstlisting}
    \end{enumerate}

%------------------------------------------------------------------------ 
%-------------- INSTALANDO A INTERFACE GRÁFICA --------------------------
%------------------------------------------------------------------------
  \subsection{Instalação da interface gráfica}
  Na instalação das interfaces, você será perguntado sobre qual mapa de teclado utilizar,
  sempre opte por ``Brazil''.
%-------------- INSTANADO LXDE ------------------------------------------
    \subsubsection{Instalando o LXDE (RECOMENDADA)}
      Recomenda-se a instalação desta interface gráfica,
      \begin{enumerate}
	\item Instalando os software da interface:
	  \begin{verbatim}
	    # apt-get install xorg lxde dbus hal obconf
	  \end{verbatim}
	\item Para fazer com que \emph{login} gráfico seja ativado:
	  \begin{verbatim}
	    # echo startlxde > ~/.xinitrc
	    # apt-get clean
	  \end{verbatim}
      \end{enumerate}
%-------------- INSTALANDO OPEN BOX -------------------------------------
    \subsubsection{Instalando o OpenBox}
      Vale observar que está interface na sua forma mais simples, é um pouco 
      ``espartana''. Contudo esta permite uma vasta possibilidade de configurações, 
      além de ser extremamente leve.
      \begin{enumerate}
	\item Instalação dos softwares e ferramentas:
	  \begin{lstlisting}
	   # apt-get install xserver-xorg-core xinit openbox xterm bzip2
	  \end{lstlisting}
	\item ao fazer o login digite:
	  \begin{lstlisting}
	   # startx
	  \end{lstlisting}
	  A principio, pode parecer que nada ocorreu. Contudo clique com o botão direito na tela.
	\item Se preferir livra-se do \emph{prompt} no login e do comando \emph{startx}, ainda no 
	  enjaulamento digite:
	  \begin{lstlisting}
	   # apt-get install gdm
	   # apt-get clean
	  \end{lstlisting}

      \end{enumerate}
%-------------- INSTALANDO GNOME ----------------------------------------
    \subsubsection{Instalando do ambiente GNOME tradicional}
      Instala a interface gráfica \emph{GNOME} tradicional. Vale observar que esta instalação 
      demora um e ocupa bastante espaço do \emph{pendrive}, contudo fornece uma interface muito 
      simples e completa.
      \begin{enumerate}
	\item Digite os seguinte comando para instalar o X:
	  \begin{lstlisting}
	   # apt-get install x-window-system
	  \end{lstlisting}
	\item Instalar o GDM:
	  \begin{lstlisting}
	   # apt-get install gdm
	  \end{lstlisting}
	\item Instalar o GNOME:
	  \begin{lstlisting}
	   # apt-get install gnome 
	   # apt-get clean
	  \end{lstlisting}
      \end{enumerate}
      A primeira inicialização pode demorar e depois de um tempo o computador reinicia. Contudo 
      os próximos \emph{boots} ficam normal.
    \subsubsection{Teste do dispositivo}
      Reinicie a sua máquina e teste o novo SO instalado no \emph{pendrive}.
%------------------------------------------------------------------------ 
%-------------------- CONFIGURANDO O XENOMAI ----------------------------
%------------------------------------------------------------------------
  \subsection{Recompilando o \emph{Kernel} e configurando o \emph{Xenomai}}
%-------------- PREPARANDO O AMBIENTE -----------------------------------
    \subsubsection{Preparando o ambiente}
      \begin{enumerate}
	\item Entre no seu sistema operacional de trabalho e conecte o \emph{pendrive}.
	\item Monte o \emph{pendrive} e faça o enjaulamento:
	  \begin{lstlisting}
	    # mount /dev/xxx /mnt
	    # chroot /mnt
	    # mount -t proc proc /proc
	  \end{lstlisting}

	\item Instalação de pacotes básico:
	  \begin{lstlisting}
	    # apt-get install kernel-package libncurses-dev fakeroot zlib1g-dev
	    # apt-get install zlibc zlib-bin
	    # apt-cache search zlib.h
	    # apt-get install zlib-bin zlib1g zlib1g-dev zlib1g-dbg libio-zlib-perl zlibc zlib-gst	
	  \end{lstlisting}

	\item Mude para o diretório de código fonte:
	  \begin{lstlisting}
	    # cd /usr/src 
	  \end{lstlisting}

	\item Realize o \emph{download} do código fonte do \emph{kernel} do \emph{Linux}:
	  \begin{lstlisting}
	    # wget http://www.kernel.org/pub/linux/kernel/v2.6/linux-2.6.32.20.tar.bz2
	  \end{lstlisting}

	\item Descomprimindo o arquivo do \emph{kernel}:
	  \begin{lstlisting}
    	    # tar -jxvf linux-2.6.32.20.tar.bz2
	  \end{lstlisting}

	\item Criar um link simbólico para o diretório do \emph{kernel}:
	  \begin{lstlisting}
	   # ln -s /usr/src/linux-2.6.32.20 linux
	  \end{lstlisting}

	\item Realizar o download do código fonte do \emph{Xenomai}:
	  \begin{lstlisting}
	   # wget http://download.gna.org/xenomai/stable/xenomai-2.5.6.tar.bz2
	  \end{lstlisting}

	\item Descompacte os arquivos do \emph{Xenomai} e modifique as permissões
	  \begin{lstlisting}
	    # tar -jxvf xenomai-2.5.6.tar.bz2 && chmod 775 -R xenomai-2.5.6
	  \end{lstlisting}

	\item Crie um link simbólico para o arquivos descompactados do \emph{Xenomai}:
	  \begin{lstlisting}
	    # ln -s /usr/src/xenomai-2.5.6 xenomai
	  \end{lstlisting}

	\item Realize o download do pacote \emph{patch} \emph{Adeos}:
	  \begin{lstlisting}
	   # wget http://download.gna.org/adeos/patches/v2.6/x86/older/adeos-ipipe-2.6.32.20-x86-2.7-03.patch
	  \end{lstlisting}

	\item Copiando o \emph{path} do \emph{Adeos} para o diretório \emph{Xenomai}
	  \begin{lstlisting}
	   # cp adeos-ipipe-2.6.32.20-x86-2.7-03.patch /usr/src/xenomai/ksrc/arch/x86/patches
	  \end{lstlisting}

	\item Aplicando o \emph{path} o no código fonte do \emph{kernel}
	  \begin{lstlisting}
	   # ./xenomai/scripts/prepare-kernel.sh --arch=x86 --adeos=/usr/src/xenomai/ksrc/arch/x86/patches/adeos-ipipe-2.6.32.20-x86-2.7-03.patch --linux=/usr/src/linux
	  \end{lstlisting}

	\item Mudando para o diretório do \emph{Linux}:
	  \begin{lstlisting}
	   # cd linux
	  \end{lstlisting}

	\item Copiando o arquivo \emph{.config} da distribuição padrão para o diretório do \emph{linux}
	  \begin{lstlisting}
	   # cp /boot/config-`uname -r` .config
	  \end{lstlisting}
      
     \subsubsection{Configurando o kernel}

	\item As configurações do \emph{kernel} são específicas para cada arquitetura que irá utilizar o novo \emph{kernel}. 
	  Desse modo, este tutorial apenas ilustra aquelas já documentadas pela literatura como sendo problemáticas 
	  com o \emph{Xenomai}.
	
	\item Digite no terminal:
	  \begin{lstlisting}
	   # make menuconfig
	  \end{lstlisting}

	\item Realize as seguinte configurações:
	  \begin{itemize}
	    \item General Setup $-->$
		\begin{itemize}
		  \item Local version – append to kernel release $--->$
		    \begin{itemize}
		      \item Digitar: -xenomai-2.5.6
		    \end{itemize}
		\end{itemize}

	      \item Power management and ACPI options $--->$
		\begin{itemize}
		  \item ACPI (Advanced Configuration and Power Interface)Support $--->$
		    \begin{itemize}
		      \item Processor
			\begin{itemize}
			  \item Desabilitar Processor
			\end{itemize}
		    \end{itemize}

		  \item APM (Advanced Power Management) BIOS Support $--->$
		    \begin{itemize}
		      \item Desabilitar APM (Advanced Power Management) BIOS Support
		    \end{itemize}

		  \item CPU Frequency scaling $--->$
		    \begin{itemize}
		      \item Desabilitar CPU Frequency scaling
		    \end{itemize}

		\end{itemize}

	      \item Processor types and features $--->$
		\begin{itemize}
		\item Escolher o processador específico. Não escolher processador genérico como "generic i586"
		\item Enable -fstack-protector buffer overflow dettector
		  \begin{itemize}
		  \item Desabilite esta opção.
		  \end{itemize}

		\item HPET Timer Support
		  \begin{itemize}
		    \item Desabilite esta opção.
		  \end{itemize}

		\item Preemption model $--->$
		  \begin{itemize}
		  \item Escolher Preemptible Kernel (Low-Latency Desktop)
		  \end{itemize}
		\end{itemize}
	      
	      \item Device Drivers $--->$
		\begin{itemize}
		\item Input device support $--->$
		  \begin{itemize}
		  \item Miscellaneous device $--->$
		    \begin{itemize}
		    \item PC Speaker support\\
		      Desabilitar esta opção
		    \end{itemize}
		  \end{itemize}
		\end{itemize}

	      \item Device Drivers $--->$
		\begin{itemize}
		  \item Debug support (Habilite esta opção)
		    \begin{itemize}
		      \item Wathdog timeout
			\begin{itemize}
			 \item Determine um valor de \emph{timeout}, recomendo (15)
			\end{itemize}

		    \end{itemize}
		\end{itemize}

	  \end{itemize}

	\item Realizando a compilação do \emph{Kernel} com \emph{framework} \emph{Xenomai}:
	  \begin{lstlisting}
	    # CONCURRENCY_LEVEL=2 CLEAN_SOURCE=no fakeroot make-kpkg --initrd kernel_image kernel_headers
	  \end{lstlisting}

	\item Subindo uma pasta para instalação do novo \emph{Kernel}:
	  \begin{lstlisting}
	  cd ..
	  \end{lstlisting}

	\item Instalando o \emph{Kernel} com o \emph{dpkg}:
	  \begin{lstlisting}
	    # dpkg -i linux-image*.deb
	  \end{lstlisting}

	\item Mudar para o diretório do \emph{Xenomai}
	  \begin{lstlisting}
	    # cd /usr/src/xenomai
	  \end{lstlisting}

	\item Executar o \emph{script configure} (opções de configuração README.INSTALL):
	  \begin{lstlisting}
	   # ./configure --enable-x86-sep --enable-smp --disable-x86-tsc
	  \end{lstlisting}

	\item Executar o comando \emph{make} e \emph{make install}
	  \begin{lstlisting}
	    # make && make install
	  \end{lstlisting}

	\item Mudar para o diretório \emph{bin} do \emph{Xenomai}
	  \begin{lstlisting}
	   # cd /usr/xenomai/bin
	  \end{lstlisting}    

      \end{enumerate}

%------------------------------------------------------------------------ 
%--------------- ENTENDENDO OS PASSOS DE INSTALAÇÃO ---------------------
%------------------------------------------------------------------------
\newpage
\section{Parte 2 - Entendendo os passos de instalação}

\newpage
\section{Bibliografia}

  \begin{lstlisting}
   http://eriberto.pro.br/wiki/index.php?title=Pendrive_de_boot_com_Debian_Lenny_customizado
   http://ubuntuforums.org/showthread.php?t=778818 
   http://www.911cd.net/forums//index.php?showtopic=21353
  \end{lstlisting}


\end{document}


%refs
%http://eriberto.pro.br/wiki/index.php?title=Pendrive_de_boot_com_Debian_Lenny_customizado
%http://ubuntuforums.org/showthread.php?t=778818
%http://www.911cd.net/forums//index.php?showtopic=21353
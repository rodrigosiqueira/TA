\documentclass[10pt, compress, aspectratio=169]{beamer}

\usetheme[numbering=fraction, progressbar=none, titleformat=smallcaps]{metropolis}
\usepackage{booktabs}
\usepackage{array}
\usepackage{listings}
\usepackage{graphicx}
\usepackage[scale=2]{ccicons}
\usepackage{url}
\usepackage{relsize}
\usepackage{wasysym}

\usepackage{pgfplots}
\usepgfplotslibrary{dateplot}

\lstset{ %
  backgroundcolor={},
  basicstyle=\ttfamily\footnotesize,
  breakatwhitespace=true,
  breaklines=true,
  captionpos=n,
  commentstyle=\color{orange},
  escapeinside={\%*}{*)},
  extendedchars=true,
  frame=n,
  keywordstyle=\color{orange},
  language=bash,
  rulecolor=\color{black},
  showspaces=false,
  showstringspaces=false,
  showtabs=false,
  numbers=left,
  numbersep=3pt,
  stepnumber=1,
  stringstyle=\color{gray},
  tabsize=2,
  keywords={thrust,plus,device_vector, copy,transform,begin,end, copyin,
  copyout, acc, \_\_global\_\_, void, int, float, main, threadIdx, blockIdx,
  blockDim, if, else, malloc, NULL, cudaMalloc, cudaMemcpy, cudaSuccess,
  cudaGetLastError, cudaDeviceSynchronize, cudaFree, cudaMemcpyDeviceToHost,
  cudaMemcpyHostToDevice, const, data, independent, kernels, loop,
  fprintf, stderr, cudaGetErrorString, EXIT_FAILURE, for, dim3},
  otherkeywords={::, \#pragma, \#include, <<<,>>>, \&, \*, +, -, /, [, ], >, <}
}

\renewcommand*{\UrlFont}{\ttfamily\smaller\relax}

\graphicspath{{images/}}

\title{Shell Variables}
\author{\footnotesize Rodrigo Siqueira \\ {\scriptsize siqueira@ime.usp.br}}
%\institute{\includegraphics[height=2cm]{imelogo}\\[0.2cm] Department of Computer Science \\ University of São Paulo}

\begin{document}

\maketitle

%------------------------------------------------------------------------------
\section{Introduction}
\begin{frame}{Overview}
  \metroset{block=fill}
  \begin{itemize}
    \item Shell Variable
    \item Environment Variable
    \item Why you should know about it?
  \end{itemize}
  \begin{alertblock}{Alert}
    There is a large number of Shell Variables, and some of them are extremely
    specific. For this part It is present just a few of them which can be
    useful in daily routines.
  \end{alertblock}
\end{frame}

%------------------------------------------------------------------------------
\section{Bourne Shell Variables}
\begin{frame}{\texttt{CDPATH}}
  \metroset{block=fill}
  \begin{exampleblock}{About}
    \begin{itemize}
      \item It directly affect the \texttt{cd} command
      \item Paths are separated by \textbf{":"}
      \item Take care with it
    \end{itemize}
  \end{exampleblock}
  Example: \\
  \texttt{export CDPATH="/home/youruser/"}\\
  \texttt{cd /tmp}\\
  \texttt{cd Downloads}
\end{frame}

\begin{frame}{\texttt{HOME}}
  \metroset{block=fill}
  \begin{exampleblock}{About}
    \begin{itemize}
      \item Keep user home path
      \item It is expand by \texttt{~}
    \end{itemize}
  \end{exampleblock}
\end{frame}

\begin{frame}{\texttt{IFS}}
  \metroset{block=fill}
  \begin{exampleblock}{About}
    \begin{itemize}
      \item It is a list of separating characters which is expanded by Bash
      \item It affects the \texttt{\$*} expansion
    \end{itemize}
  \end{exampleblock}
  If you want to see the values inside \texttt{\$IFS}, just try: \\
  \texttt{echo "\$IFS" | cat -ETv}
\end{frame}

\begin{frame}{\texttt{OPTARG} and \texttt{OPTIND}}
  \metroset{block=fill}
  \begin{exampleblock}{About}
    \begin{itemize}
      \item Both variables are used by \texttt{getopts} builtin
      \item \texttt{OPTARG}: The value captured from \texttt{getopts}
      \item \texttt{OPTIND}: The index captured from \texttt{getopts}
    \end{itemize}
  \end{exampleblock}
\end{frame}

%------------------------------------------------------------------------------
\section{Bash Variables}

\begin{frame}{\texttt{BASH}, \texttt{SHELL}, and \texttt{BASH\_COMMAND}}
  \metroset{block=fill}
  \begin{exampleblock}{About}
    \begin{itemize}
      \item \texttt{BASH}: Path of the current Bash running
      \item \texttt{SHELL}: Path of the default Shell
      \item \texttt{BASH\_COMMAND}: It has the current command or about to be
            executed
    \end{itemize}
  \end{exampleblock}
\end{frame}

\begin{frame}{\texttt{EUID}, \texttt{UID}, \texttt{GROUPS}, and \texttt{HOSTNAME}}
  \metroset{block=fill}
  \begin{exampleblock}{About}
    \begin{itemize}
      \item \texttt{EUID}: Shows the \textit{effective} id from the current user
      \item \texttt{UID}: Shows the \textit{real} id from the current user
      \item \texttt{GROUPS}: Shows a list of id groups wherein the current user
            belongs
      \item \texttt{HOSTNAME}: Shows the machine hostname
    \end{itemize}
  \end{exampleblock}

  \begin{exampleblock}{Real vs Effective user id}
    \begin{itemize}
      \item \textbf{real}: Who you really are
      \item \textbf{effective}: What the OS looks to make a decision
    \end{itemize}
  \end{exampleblock}
\end{frame}

\begin{frame}{\texttt{FUNCNAME} and \texttt{LINENO}}
  \metroset{block=fill}
  \begin{exampleblock}{About}
    \begin{itemize}
      \item \texttt{FUNCNAME}: Only works inside function, and retrieves
            function names current in the stack.
      \item \texttt{LINENO}: Shows current line executed by the Shell
    \end{itemize}
  \end{exampleblock}
\end{frame}

\begin{frame}{\texttt{FUNCNAME} and \texttt{LINENO}}
  \metroset{block=fill}
  \lstinputlisting{codes/funcname_lineno.sh}
\end{frame}

\begin{frame}{\texttt{HISTFILE}, and \texttt{HISTFILESIZE}}
  \metroset{block=fill}
  \begin{exampleblock}{About}
    \begin{itemize}
      \item \texttt{HISTFILE}: Path to the history file
      \item \texttt{HISTFILESIZE}: Maximum number accepted n the history file
    \end{itemize}
  \end{exampleblock}
\end{frame}

\begin{frame}{\texttt{SECONDS}}
  \metroset{block=fill}
  \begin{exampleblock}{About}
    \begin{itemize}
      \item Expands to the number of seconds since the shell was started
    \end{itemize}
  \end{exampleblock}
\end{frame}

\begin{frame}{\texttt{SECONDS}}
  \metroset{block=fill}
  \lstinputlisting{codes/seconds_example.sh}
\end{frame}

\begin{frame}{\texttt{RANDOM}}
  \metroset{block=fill}
  \begin{exampleblock}{About}
    \begin{itemize}
      \item Each time this variable is referred, it produces a random number
      \item Write a value to this variable change the seed
    \end{itemize}
  \end{exampleblock}
\end{frame}

\begin{frame}{\texttt{RANDOM}}
  \metroset{block=fill}
  \lstinputlisting{codes/random_example.sh}
\end{frame}

%------------------------------------------------------------------------------
\section{About this presentation}
\begin{frame}[standout]
  % TODO: Improve it
   \begin{center}\ccbysa\end{center}
\end{frame}

%TODO: Bibliography
% Real vs Effective id: https://en.wikipedia.org/wiki/User_identifier#Saved_user_ID
\maketitle

\end{document}

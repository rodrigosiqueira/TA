\documentclass[10pt, compress, aspectratio=169]{beamer}

\usetheme[numbering=fraction, progressbar=none, titleformat=smallcaps]{metropolis}
\usepackage{booktabs}
\usepackage{array}
\usepackage{listings}
\usepackage{graphicx}
\usepackage[scale=2]{ccicons}
\usepackage{url}
\usepackage{relsize}
\usepackage{wasysym}

\usepackage{pgfplots}
\usepgfplotslibrary{dateplot}

\lstset{ %
  backgroundcolor={},
  basicstyle=\ttfamily\footnotesize,
  breakatwhitespace=true,
  breaklines=true,
  captionpos=n,
  commentstyle=\color{orange},
  escapeinside={\%*}{*)},
  extendedchars=true,
  frame=n,
  keywordstyle=\color{orange},
  language=bash,
  rulecolor=\color{black},
  showspaces=false,
  showstringspaces=false,
  showtabs=false,
  numbers=left,
  numbersep=3pt,
  stepnumber=1,
  stringstyle=\color{gray},
  tabsize=2,
  keywords={thrust,plus,device_vector, copy,transform,begin,end, copyin,
  copyout, acc, \_\_global\_\_, void, int, float, main, threadIdx, blockIdx,
  blockDim, if, else, malloc, NULL, cudaMalloc, cudaMemcpy, cudaSuccess,
  cudaGetLastError, cudaDeviceSynchronize, cudaFree, cudaMemcpyDeviceToHost,
  cudaMemcpyHostToDevice, const, data, independent, kernels, loop,
  fprintf, stderr, cudaGetErrorString, EXIT_FAILURE, for, dim3},
  otherkeywords={::, \#pragma, \#include, <<<,>>>, \&, \*, +, -, /, [, ], >, <}
}

\renewcommand*{\UrlFont}{\ttfamily\smaller\relax}

\graphicspath{{images/}}

\title{Builtin commands}
\author{\footnotesize Rodrigo Siqueira \\ {\scriptsize siqueira@ime.usp.br}}
%\institute{\includegraphics[height=2cm]{imelogo}\\[0.2cm] Department of Computer Science \\ University of São Paulo}

\begin{document}

\maketitle

%------------------------------------------------------------------------------
\section{Introduction}
\begin{frame}{Overview}
  % TODO: Create this section
  \begin{itemize}
    \item Computers...
  \end{itemize}
\end{frame}

%------------------------------------------------------------------------------
\section{Bourne Shell Builtin}
\begin{frame}{Colon (\texttt{:})}
  \metroset{block=fill}
  \begin{alertblock}{Syntax}
    \texttt{:} (a colon)
  \end{alertblock}

  \begin{exampleblock}{About}
    \begin{itemize}
      \item \textbf{Do nothing! Just it! :)}
      \item You can use it as \textbf{true}
      \item Special built-in
    \end{itemize}
  \end{exampleblock}
\end{frame}

\begin{frame}{Period (.)}
  \metroset{block=fill}
  \begin{alertblock}{Syntax}
    \texttt{. filename [arguments]} (a period)
  \end{alertblock}

  \begin{exampleblock}{About}
    \begin{itemize}
      \item \textbf{Read and execute commands from the \texttt{filename}}
      \item It uses \texttt{PATH} to find the file
      \item You can do some useful things with it :)
    \end{itemize}
  \end{exampleblock}

  Using \texttt{.} to include another script

  \lstinputlisting{codes/include_example.sh}
\end{frame}

\begin{frame}{\texttt{exit} and \texttt{return}}
  \metroset{block=fill}
  \begin{alertblock}{Syntax}
    \texttt{exit [n]}
  \end{alertblock}

  \metroset{block=fill}
  \begin{alertblock}{Syntax}
    \texttt{return [n]}
  \end{alertblock}
\end{frame}

\begin{frame}{\texttt{break}}
  \metroset{block=fill}
  \begin{alertblock}{Syntax}
    \texttt{break [n]}
  \end{alertblock}

  \begin{exampleblock}{About}
    \begin{itemize}
      \item Exit from: \texttt{for}, \texttt{while},
            \texttt{until}, and \texttt{select}
      \item \texttt{n} must be greater than or equal to 1
    \end{itemize}
  \end{exampleblock}
\end{frame}

\begin{frame}{\texttt{break}}
  Example:
  \lstinputlisting{codes/break_with_n.sh}
\end{frame}

\begin{frame}{\texttt{continue}}
  \metroset{block=fill}
  \begin{alertblock}{Syntax}
    \texttt{continue [n]}
  \end{alertblock}

  \begin{exampleblock}{About}
    \begin{itemize}
      \item \textbf{Resume} the next iteration
      \item \texttt{n} must be greater than or equal to 1
      \item \textbf{Return}: Return status is zero unless \texttt{n} is not
            greater than or equal to 1
    \end{itemize}
  \end{exampleblock}
\end{frame}

\begin{frame}{\texttt{continue}}
  Example:
  \lstinputlisting{codes/continue_with_n.sh}
\end{frame}

\begin{frame}{\texttt{cd}}
  \metroset{block=fill}
  \begin{alertblock}{Syntax}
    \texttt{cd [-L|[-P[-e]] [-@] [diretory]}
  \end{alertblock}

  \begin{exampleblock}{About}
    \begin{itemize}
      \item In a few words: \textbf{C}hange \textbf{D}irectory :)
      \item Some cool stuffs with \texttt{cd}:
        \begin{enumerate}
          \item Go to \texttt{\$HOME}: \texttt{cd}
          \item Go to last directory (\texttt{\$OLDPWD}): \texttt{cd -}
          \item Go to upper directory: \texttt{cd ..}
        \end{enumerate}
    \end{itemize}
  \end{exampleblock}
\end{frame}

\begin{frame}{\texttt{eval}}
  \metroset{block=fill}
  \begin{alertblock}{Syntax}
    \texttt{eval [arguments]}
  \end{alertblock}

  \begin{exampleblock}{About}
    \begin{itemize}
      \item Concatenate arguments into a single command
    \end{itemize}
  \end{exampleblock}

  \lstinputlisting{codes/eval_example.sh}
\end{frame}

\begin{frame}{\texttt{exec}}
  \metroset{block=fill}
  \begin{alertblock}{Syntax}
    \texttt{exec [-cl] [-a name] [command] [arguments]}
  \end{alertblock}
  \lstinputlisting{codes/exec_input.sh}
\end{frame}

\begin{frame}{\texttt{export}}
  \metroset{block=fill}
  \begin{alertblock}{Syntax}
    \texttt{export [-fn] [-p] [name[=value]]}
  \end{alertblock}

  \begin{exampleblock}{About}
    \begin{itemize}
      \item Mark name to e passed to child processes in the environment
    \end{itemize}
  \end{exampleblock}

\end{frame}

\begin{frame}{\texttt{getopts}}
  \metroset{block=fill}
  \begin{alertblock}{Syntax}
    \texttt{getopts "options" name [arguments]}
  \end{alertblock}

  \begin{exampleblock}{About}
    \begin{itemize}
      \item If \texttt{options} has ':' than a value is expected
      \item \texttt{\$OPTARG} has the value parsed from ':'
      \item \texttt{?} is set to \texttt{\$OPTARG} if option is unknown
    \end{itemize}
  \end{exampleblock}
\end{frame}

\begin{frame}{\texttt{getopts}}
  Example:
  \metroset{block=fill}
  \lstinputlisting{codes/getopts_simple.sh}
\end{frame}

\begin{frame}{\texttt{hash}, \texttt{readonly}, \texttt{pwd}, and \texttt{time}}

  \metroset{block=fill}
  \begin{alertblock}{Syntax}
    \texttt{hash [-r] [-p file] [-dt] [name]}
  \end{alertblock}

  \begin{alertblock}{Syntax}
    \texttt{readonly [-aAf] [-p] [name[=value]] ...}
  \end{alertblock}

  \begin{alertblock}{Syntax}
    \texttt{pwd [-LP]}
  \end{alertblock}

  \begin{alertblock}{Syntax}
    \texttt{time}
  \end{alertblock}
\end{frame}

\begin{frame}{\texttt{shift}}
  \metroset{block=fill}
  \begin{alertblock}{Syntax}
    \texttt{shift [n]}
  \end{alertblock}

  \begin{exampleblock}{About}
    \begin{itemize}
      \item Shift on \textit{Positional Parameters}
    \end{itemize}
  \end{exampleblock}
\end{frame}

\begin{frame}{\texttt{test}}
  \metroset{block=fill}
  \begin{alertblock}{Syntax}
    \texttt{test expression}
  \end{alertblock}
\end{frame}

\begin{frame}{\texttt{trap}}
  \metroset{block=fill}
  \begin{alertblock}{Syntax}
    \texttt{trap [-lp] [arguments] [sigspec...]}
  \end{alertblock}

  \begin{exampleblock}{About}
    \begin{itemize}
      \item Execute \texttt{arguments} if \texttt{sigspec} is received
      \item Useful: \texttt{trap -l}
      \item Two important \texttt{sigspec}: \texttt{EXIT} and \texttt{ERR}
    \end{itemize}
  \end{exampleblock}

  \lstinputlisting{codes/trap_simple.sh}
\end{frame}

%------------------------------------------------------------------------------
\section{Bash builtin}
\begin{frame}{\texttt{alias} and \texttt{unalias}}
  \metroset{block=fill}
  \begin{alertblock}{Syntax}
    \texttt{alias [-p] [name[=value]...]}
  \end{alertblock}

  \begin{alertblock}{Syntax}
    \texttt{unalias [-a] [name ...]}
  \end{alertblock}

  \begin{exampleblock}{About}
    \begin{itemize}
      \item Create and remove an alias
      \item Useful: \texttt{alias -p}
    \end{itemize}
  \end{exampleblock}
  Example: \\
  \texttt{alias ls='ls --color=auto'} \\
  \texttt{unalias ls} \\
\end{frame}

\begin{frame}{\texttt{bind}}
  \metroset{block=fill}
  \begin{alertblock}{Syntax}
    \texttt{bind [-m keymap] [-lpsvPSVX]}
  \end{alertblock}

  \begin{exampleblock}{About}
    \begin{itemize}
      \item Allow bind a key to a command
    \end{itemize}
  \end{exampleblock}
  Example: \\
  \texttt{bind '"\textbackslash{e}[24\textasciitilde":"pwd\textbackslash{n}"'}
\end{frame}

\begin{frame}{\texttt{builtin} and \texttt{command}}
  \metroset{block=fill}
  \begin{alertblock}{Syntax}
    \texttt{builtin [shell-builtin [arguments]]}
  \end{alertblock}

  \begin{exampleblock}{About}
    If you have a function with the same name of a build in, your
    function will be called first. If you want to explicit call the build in,
    use \texttt{buildin} command
  \end{exampleblock}

  \begin{alertblock}{Syntax}
    \texttt{command [-pVv] command [arguments ...]}
  \end{alertblock}

  \begin{exampleblock}{About}
    \begin{itemize}
      \item Basically, runs \texttt{command}
      \item Only executes command found in the \texttt{PATH} or buildin
    \end{itemize}
  \end{exampleblock}
\end{frame}

\begin{frame}{\texttt{caller}}
  \metroset{block=fill}
  \begin{alertblock}{Syntax}
    \texttt{caller [expression]}
  \end{alertblock}

  \begin{exampleblock}{About}
    Prints the execution frame (it is related to stack frame concept)
  \end{exampleblock}

  \lstinputlisting{codes/caller_test.sh}
\end{frame}

\begin{frame}{\texttt{declare}}
  \metroset{block=fill}
  \begin{alertblock}{Syntax}
    \texttt{declare [-aAfFgilnrtux] [-p] [name[=value]...]}
  \end{alertblock}

  \begin{exampleblock}{About}
    Set attributes to variables
  \end{exampleblock}
\end{frame}

\begin{frame}{\texttt{declare}}
  \metroset{block=fill}
  \lstinputlisting{codes/declare_ref.sh}
\end{frame}

\begin{frame}{\texttt{echo}}
  \metroset{block=fill}
  \begin{alertblock}{Syntax}
    \texttt{echo [-neE] [arguments...]}
  \end{alertblock}
\end{frame}

\begin{frame}{\texttt{enable}}
  \metroset{block=fill}
  \begin{alertblock}{Syntax}
    \texttt{enable [-a] [-dnps] [-f filename] [name...]}
  \end{alertblock}

  \begin{exampleblock}{About}
    Enables and disable buildin commands
  \end{exampleblock}
\end{frame}

\begin{frame}{\texttt{help}}
  \metroset{block=fill}
  \begin{alertblock}{Syntax}
    \texttt{help [-dms] [pattern]}
  \end{alertblock}

  \begin{exampleblock}{About}
    This help refers to buildin commands
  \end{exampleblock}
\end{frame}

\begin{frame}{\texttt{let}}
  \metroset{block=fill}
  \begin{alertblock}{Syntax}
    \texttt{let expression [expression...]}
  \end{alertblock}

  \begin{exampleblock}{About}
    Arithmetic to be performed on shell variable
  \end{exampleblock}

  Example: \\
  \texttt{let "a = 3 + 20"}
\end{frame}

\begin{frame}{\texttt{local}}
  \metroset{block=fill}
  \begin{alertblock}{Syntax}
    \texttt{local [option] name[=value]}
  \end{alertblock}

  \begin{exampleblock}{About}
    \begin{itemize}
      \item Make variable available only inside function
      \item Only used inside function
      \item Accepts the same parameters of declare
    \end{itemize}
  \end{exampleblock}
\end{frame}

\begin{frame}{\texttt{logout}}
%TODO
\end{frame}

\begin{frame}{\texttt{mapfile}}
%TODO
\end{frame}

\begin{frame}{\texttt{printf}}
  \metroset{block=fill}
  \begin{alertblock}{Syntax}
    \texttt{printf [-v var] format [args]}
  \end{alertblock}
  Examples: \\
  \texttt{printf "\%x\textbackslash{n}" 15} \\
  \texttt{printf -v new "\%x\textbackslash{n}" 15}
\end{frame}

\begin{frame}{\texttt{read}}
%TODO
\end{frame}

\begin{frame}{\texttt{readarray}}
\end{frame}

\begin{frame}{\texttt{source}}
\end{frame}

\begin{frame}{\texttt{type}}
  \metroset{block=fill}
  \begin{alertblock}{Syntax}
    \texttt{type [-afptP] [name]}
  \end{alertblock}
\end{frame}

\begin{frame}{\texttt{typeset}}
% TODO
\end{frame}

\begin{frame}{\texttt{ulimit}}
  \metroset{block=fill}
  \begin{alertblock}{Syntax}
    \texttt{ulimit [-HSabcdefiklmnpqrstuvxPT] [limit]}
  \end{alertblock}

  \begin{exampleblock}{About}
    \begin{itemize}
      \item Provides control over the resources available to processes
      \item See total of soft limit assotiated with a resource:
            \texttt{ulimit -S}
      \item \textbf{hard-limit}: Only root can change the limit
      \item \textbf{soft-limit}: User can change the limit
    \end{itemize}
  \end{exampleblock}
\end{frame}

%------------------------------------------------------------------------------
\section{About this presentation}
\begin{frame}[standout]
  % TODO: Improve it
   \begin{center}\ccbysa\end{center}
\end{frame}

%TODO: Bibliography
% break [n]: http://tldp.org/LDP/abs/html/loopcontrol.html
% continue [n]: http://tldp.org/LDP/abs/html/loopcontrol.html
% exec: http://wiki.bash-hackers.org/commands/builtin/exec
% caller: http://wiki.bash-hackers.org/commands/builtin/caller
% key num code: http://invisible-island.net/xterm/xterm-function-keys.html
\maketitle

\end{document}

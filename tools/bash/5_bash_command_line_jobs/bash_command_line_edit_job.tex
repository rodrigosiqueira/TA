\documentclass[10pt, compress, aspectratio=169]{beamer}

\usetheme[numbering=fraction, progressbar=none, titleformat=smallcaps]{metropolis}
\usepackage{booktabs}
\usepackage{array}
\usepackage{listings}
\usepackage{graphicx}
\usepackage[scale=2]{ccicons}
\usepackage{url}
\usepackage{relsize}
\usepackage{wasysym}

\usepackage{pgfplots}
\usepgfplotslibrary{dateplot}

\lstset{ %
  backgroundcolor={},
  basicstyle=\ttfamily\footnotesize,
  breakatwhitespace=true,
  breaklines=true,
  captionpos=n,
  commentstyle=\color{orange},
  escapeinside={\%*}{*)},
  extendedchars=true,
  frame=n,
  keywordstyle=\color{orange},
  language=bash,
  rulecolor=\color{black},
  showspaces=false,
  showstringspaces=false,
  showtabs=false,
  numbers=left,
  numbersep=3pt,
  stepnumber=1,
  stringstyle=\color{gray},
  tabsize=2,
  keywords={thrust,plus,device_vector, copy,transform,begin,end, copyin,
  copyout, acc, \_\_global\_\_, void, int, float, main, threadIdx, blockIdx,
  blockDim, if, else, malloc, NULL, cudaMalloc, cudaMemcpy, cudaSuccess,
  cudaGetLastError, cudaDeviceSynchronize, cudaFree, cudaMemcpyDeviceToHost,
  cudaMemcpyHostToDevice, const, data, independent, kernels, loop,
  fprintf, stderr, cudaGetErrorString, EXIT_FAILURE, for, dim3},
  otherkeywords={::, \#pragma, \#include, <<<,>>>, \&, \*, +, -, /, [, ], >, <}
}

\renewcommand*{\UrlFont}{\ttfamily\smaller\relax}

\graphicspath{{images/}}

\title{Command Line Editing and Jobs}
\author{\footnotesize Rodrigo Siqueira \\ {\scriptsize siqueira@ime.usp.br}}
%\institute{\includegraphics[height=2cm]{imelogo}\\[0.2cm] Department of Computer Science \\ University of São Paulo}

\begin{document}

\maketitle

%------------------------------------------------------------------------------
\section{Introduction}
\begin{frame}{Basic definitions}
  \metroset{block=fill}
  \begin{exampleblock}{Bases}
    \begin{itemize}
      \item C-(k): C means Ctrl and \textit{k} mean any key
      \item M-(k): M means Meta character, usually, it is map to the ALT key
        \begin{itemize}
          \item If your ALT key is not map to M, you can type ESC then the key
          \item You can map M to ALT in your system. Sometimes, you just have
                to add \texttt{xterm*metaSendsEscape: true} in your
                \texttt{.Xdefaults}
        \end{itemize}
      \item Bash use \textbf{Readline} library to handle the interaction, as a
            result, it supports many useful commands for manipulating the text
            you type in it.
    \end{itemize}
  \end{exampleblock}
\end{frame}

%------------------------------------------------------------------------------
\section{Essentials commands}

\begin{frame}{Essentials}
  \metroset{block=fill}
  \begin{exampleblock}{Commands}
    \begin{itemize}
      \item \texttt{C-b}: Move one character backward
      \item \texttt{C-f}: Move one character forward
      \item \texttt{C-d}: Delete the character under the cursor
      \item \texttt{C-u}: Clean the line
      \item \texttt{C-a}: Move the cursor to the beginning
      \item \texttt{C-e}: Move the cursor to the end of the line
      \item \texttt{M-f}: Delete one \textit{word} forward
      \item \texttt{M-b}: Delete one \textit{word} backward
      \item \texttt{C-l}: Clean the screen
    \end{itemize}
  \end{exampleblock}
\end{frame}

\begin{frame}{Readline Killing Commands (CUT and PASTE)}
  \metroset{block=fill}
  \begin{exampleblock}{Commands}
    \begin{itemize}
      \item \texttt{C-k}: Cut from the cursor to the end of the line
      \item \texttt{M-d}: Cut from the cursor forward one word
      \item \texttt{C-w}: Cut from the cursor backward one word
      \item \texttt{C-y}: Paste
      \item \texttt{M-y}: Rotate the \textit{kill-ring}
    \end{itemize}
  \end{exampleblock}
\end{frame}

\begin{frame}{Searching for commands in the history}
  \metroset{block=fill}
  \begin{exampleblock}{Commands}
    \begin{itemize}
      \item \texttt{C-r}: Search on history
      \item \texttt{C-j}: Abort the search
      \item \texttt{C-g}: Abort the search and restore the previous command
    \end{itemize}
  \end{exampleblock}
\end{frame}
%------------------------------------------------------------------------------
\section{Readline Files}

\begin{frame}{Basic files to configure Readline}
  \metroset{block=fill}
  \begin{exampleblock}{About basic files}
    \begin{itemize}
      \item Readline reads from two files in the following order:
        \begin{enumerate}
          \item \texttt{/etc/inputrc}
          \item \texttt{~/.inputrc}
        \end{enumerate}
      \item \texttt{C-x C-r}: Reloads inputrc files
      \item Inputrc files has a basic syntax
    \end{itemize}
  \end{exampleblock}
\end{frame}

\begin{frame}{Inputrc file syntax}
  \metroset{block=fill}
  \begin{exampleblock}{Syntax}
    \begin{itemize}
      \item Comments: Lines beinning with a \texttt{\#}
      \item Set variable: \texttt{set variable value}
      \begin{enumerate}
        \item \texttt{variable}: Variables name are defined by the Readline.
              There is a numerous option for configure many different aspects
              of the Readline. Take a look at the documentation to see all the
              options
        \item \texttt{value}: Each variable can define a different value,
              in the documentation you can find the possible values per variable
      \end{enumerate}
      \item Key Binding:
        \begin{enumerate}
          \item \texttt{keyname: function-name}
          \item \texttt{"keyseq": function-name}
          \item Example: \texttt{\textbackslash{Control}-o: "dude..."}
          \item Example: \texttt{"\textbackslash{C}-x\textbackslash{C}-r": "hey dude..."}
        \end{enumerate}
      \item Conditional: \texttt{\textdollar{}if
            \textit{compare} ... \textdollar{}endif}
    \end{itemize}
  \end{exampleblock}
\end{frame}

%------------------------------------------------------------------------------
\section{Job Control}

\begin{frame}{Basic files to configure Readline}
  \metroset{block=fill}
  \begin{exampleblock}{Basic}
    \begin{itemize}
      \item Job control refers to the ability to selectively stop and continue
            the process execution
      \item Shell keep a table of processes
      \item ID group
      \begin{itemize}
        \item \textbf{Foreground}
        \item \textbf{Background}
      \end{itemize}
    \end{itemize}
  \end{exampleblock}
\end{frame}

\begin{frame}{Job Control Builtins and commands}
  \metroset{block=fill}
  \begin{exampleblock}{Basic}
    \begin{itemize}
      \item \texttt{C-z}: Put the current process in background
      \item \texttt{\%}: Get the last processes put in background
      \item \texttt{\%n}: Get the process with the number n
      \item \texttt{\%-}: Current job
      \item \texttt{bg [jobspec ...]}: Resume each suspended jobspec
      \item \texttt{fg [jobspec]}: Resume the jobspec
      \item \texttt{jobs [-lnprs] [jobspec]}: List active jobs
      \item \texttt{kill [-s sigspec] [-n signum] [-sigspec] jobspec or pid}
      \item \texttt{wait [-n] [jobspec or pid ...]}
      \item \texttt{disown [-ar] [-h] [jobspec ... | pid ...]}
      \item \texttt{suspend [-f]}
    \end{itemize}
  \end{exampleblock}
\end{frame}

%------------------------------------------------------------------------------
\section{About this presentation}
\begin{frame}[standout]
  % TODO: Improve it
   \begin{center}\ccbysa\end{center}
\end{frame}

%TODO: Bibliography
% break [n]: http://tldp.org/LDP/abs/html/loopcontrol.html
% continue [n]: http://tldp.org/LDP/abs/html/loopcontrol.html
% exec: http://wiki.bash-hackers.org/commands/builtin/exec
% caller: http://wiki.bash-hackers.org/commands/builtin/caller
% key num code: http://invisible-island.net/xterm/xterm-function-keys.html
\maketitle

\end{document}

\documentclass{beamer}

\usepackage[utf8x]{inputenc}
\usepackage{default}
%Configurações gerais
\usetheme{PaloAlto}
\usecolortheme{seahorse}

\title{Sistemas Digitais 2\\ \textbf{Introdução a Máquina de Estados Finita} \\ \textbf{L. Filomeno  Fernandes}}
\date{Brasíla, 02/2011}
\institute{\textbf{Universidade de Brasília - Faculdade do Gama}} 

\begin{document}

%SLIDE INICIAL DE APRESENTAÇÃO
\begin{frame}
  \titlepage
\end{frame}

%SLIDES == INTRODUÇÃO
\section{Objetivos}
\begin{frame}
  \frametitle{Máquinas de estado $\xrightarrow{}$ Objetivos do capítulo}
  \begin{itemize}
   \item Descrever os componentes de uma máquina de estados.
   \item Distinguir implementações entre máquinas de estado Moore e Mealy.
   \item Desenhar o diagrama de máquinas de estados a partir de uma descrição verbal.
   \item Usar a projeto de máquina de estado clássico (tabela de estados) para determinar as equações Booleanas das máquinas de estados.
   \item Transladar (Traduzir) as equações Booleanas da máquina de estados para um desenho gráfico.
   \item Escrever o código VHDL para implementar máquinas de estados.
  \end{itemize}
\end{frame}

%SLIDE == CONCEITOS FUNDAMENTAIS
\section{Conceitos fundamentais}
\begin{frame}
 \frametitle{Conceitos fundamentais}
  \begin{itemize}
   \item Circuitos combinacionais: para cada combinação de entradas obtêm-se sempre a mesma saída.
   \begin{itemize}
    \item Álgebra de Boole 
    \item Tabela verdade
    \item Mapas de Veitch-Karnaugh
   \end{itemize}
  \end{itemize}
\end{frame}

\begin{frame}
 \frametitle{Conceitos fundamentais}
 \begin{itemize}
  \item Circuitos seqüências: em que a saída depende da combinação de entrada atual bem como das entrada anteriores.
  \begin{itemize}
   \item Circuitos com memória, dispositivos que permitem o armazenamento de valores (0 ou 1) em suas saídas por um certo tempo mediante o uso de flip-flops
   \item Autômatos ou máquinas de estado finitas
   \item Diagramas de estado
  \end{itemize}
 \end{itemize}
\end{frame}

%SLIDE == MÁQUINAS DE ESTADO
\section{Máquinas de estado}
\begin{frame}
 \frametitle{Máquinas de estado}
 \begin{itemize}
  \item Definição\pause
  \item Estrutura\pause
  \item Classificação\pause
  \item Maquinas de estado síncronas \pause
  \begin{itemize}
   \item Análise \pause
   \item Projeto 
  \end{itemize}
 \end{itemize}
\end{frame}

%SLIDE == SIMPLIFICAÇÃO DE UM COMPUTADOR MODERNO
\section{Definição}
\begin{frame}
 \frametitle{Definição}
  Antes de anunciar a definição realizamos uma análise, por meio de uma simplificação de um computador moderno.
\end{frame}

  \begin{frame}
   \frametitle{Simplificação de um computador moderno }
   \begin{itemize}
    \item O computador armazena informações na forma binária.\pause
    \item A qualquer momento, o computador contém determinadas informações, de forma que sua memória interna contém algum padrão de 
	  dígitos binários, que chamamos de \textbf{estados} do computador nesse momento. \pause
    \item O computador contém uma quantidade finita de memória, exite um número finito (apesar de grande) de diferentes estados que 
	  ele pode assumir.\pause
    \item Um clock interno sincroniza as ações do computador. Em um ciclo de clock, a entrada pode ser lida, o que pode mudar algumas 
	  das posições de memória e, portanto, mudar o estado da máquina para um novo estado.
   \end{itemize}
  \end{frame}

  \begin{frame}
   \frametitle{Simplificação de um computador moderno }
   \begin{itemize}
    \item O que o novo estado representa depende da entrada, bem como do estado anterior. SE ESSES DOIS FATORES FOREM CONHECIDOS, A 
	  ALTERAÇÃO É PREVISÍVEL E NÃO ALEATÓRIA.\pause
    \item O conteúdo de certas células são disponíveis como saída, o estado da máquina determina a sua saída. Dessa forma, ao cabo de 
	  uma sucessão de ciclos de clock, a máquina produz uma sequencia de saídas em resposta a uma sequencia de entradas.
   \end{itemize}
  \end{frame}

%SLIDE == ANALISE
  \begin{frame}
   \frametitle{Analisando}
	Antes de apresentar a definição formal de uma Máquina de Estados Finita, veja as cinco características fundamentais das máquinas de 
	estado e relacione-as com o computador digital descrita anteriormente.\pause
	
      \begin{block}{Características}
	\begin{itemize}
	  \item As operações da máquina são \textbf{sincronizadas} por ciclos discretos do clock.
	  \item A máquina procede de uma forma \textbf{determinística}, isto é, suas ações em resposta a uma sequencia de entrada são completamente 
		previsíveis.
	  \item A máquina responde a \textbf{entradas}.
	\end{itemize}
      \end{block}
  \end{frame}

  \begin{frame}
   \frametitle{Analisando}
       \begin{block}{Características}
	\begin{itemize}
	\item Existe um \textbf{número finito de estados} que a máquina pode alcançar. A qualquer momento, a máquina está em exatamente um desses 
	      estados. Qual o estado que ela estará a seguir é uma função do estado atual e da entrada atual. O estado atual, no entanto, depende 
	      dos estados e entradas anteriores, enquanto que o estado anterior depende de seus estados e entradas anteriores, e assim por diante, 
	      até chegar de volta a configuração inicial. Portanto, o estado da máquina a qualquer momento serve como uma espécie de memória das 
	      entradas anteriores.\pause
	 \item A máquina é capaz de produzir saídas. A natureza da saída é uma função do estado atual da máquina, o que significa que também depende 
	      das entradas anteriores. 
	\end{itemize}
      \end{block}
  \end{frame}

%SLIDE == DEFINIÇÃO
  \begin{frame}
   \frametitle{Definição}
       \begin{block}{\textbf{Definição}}
	  M = $[S, I, O, f_s, f_o]$ é uma \textbf{máquina de estado finito} se S for um conjunto de estados, I for um conjunto finito de símbolos de entrada (o 
	  alfabeto de entrada), O for o conjunto finito de símbolos de saída (o alfabeto de saída) e $f_s$ e $f_o$ forem funções onde $f_s: S\times I \xrightarrow{} S$ e 
	  $f_o: S \xrightarrow{} O$. A máquina é sempre iniciada a fim de começar em um estado inicial fixo $s_o$. 
      \end{block}
  \end{frame}

  \begin{frame}
   \frametitle{Definição}
       \includegraphics[height=2.7in, width=4in]{entendendo_definicao_1.png}
  \end{frame}

  \begin{frame}
   \frametitle{Definição}
       \includegraphics[height=2.7in, width=4in]{entendendo_definicao_2.png}
  \end{frame}

  \begin{frame}
   \frametitle{Definição}
       \includegraphics[height=2.7in, width=4in]{entendendo_definicao_3.png}
  \end{frame}

  \begin{frame}
   \frametitle{Definição}
       \includegraphics[height=2.7in, width=4in]{entendendo_definicao_4.png}
  \end{frame}

  \begin{frame}
   \frametitle{Definição}
       \includegraphics[height=2.7in, width=4in]{entendendo_definicao_5.png}
  \end{frame}

  \begin{frame}
   \frametitle{Definição}
       \includegraphics[height=2.7in, width=4in]{entendendo_definicao_6.png}
  \end{frame}

%SEÇÃO == Termos fundamentais
\section{Estrutura}
  \begin{frame}
    \frametitle{Estrutura}
    \begin{itemize}
      \item Suponha que queiramos que um dado circuito sequencial realize cinco operações distintas onde cada operação representará um estado, 
	    portanto teremos cinco estados (um para cada opção). Além disto cada operação pode ser realizada milhares de vezes. \pause
      \item Outra característica deste circuito é que ele mesmo é responsável por ir de um estado atual para o próximo, ou seja, ele muda para 
	    o próximo estado lógico do circuito (next-state logic circuit).\pause
      \item O próximo estado nada mais é que um circuito combinacional que recebe o conteúdo de um estado da memória e as entradas atuais.\pause
      \item As saídas do próximo estado lógico, são usados para mudar o conteúdo do estado da memória (flip-flops). O circuito muda de estado quando 
	    o conteúdo da memória de estado muda.
    \end{itemize}
 \end{frame}

  \begin{frame}
    \frametitle{Estrutura}
    \includegraphics[height=1.3in, width=4in]{modelo_1.png}
  \end{frame}

  \begin{frame}
    \frametitle{Estrutura}
    \includegraphics[height = 1.3in, width = 4in]{modelo_2.png}
  \end{frame}

  \begin{frame}
    \frametitle{Estrutura}
    As saídas são dependentes do estado anterior, desta forma um estado é usado para lembrar as entradas anteriores e também determinar as saídas geradas. 
    \includegraphics[height = 1.3in, width = 4in]{modelo_3.png}
  \end{frame}

  \begin{frame}
    \frametitle{Estrutura}
    \begin{itemize}
      \item Um circuito sequencial também é conhecido por máquina de estados finitos (finite-state machine FSM), devido ao tamanho da memória ser finita, 
	    portanto o número de diferentes estados também é finito.\pause
      \item Este tipo de circuito é utilizado em projetos de unidades de controle, devido a sua flexibilidade e a possibilidade de haver várias saídas.
    \end{itemize}
  \end{frame}

%SEÇÃO == MODELOS DE FSM
\section{Classificação de FSM}
\begin{frame}
  \frametitle{Classificação de FSM}
   Toda máquina de estado possuí três estruturas básicas, são elas:
\end{frame}

\begin{frame}
  \frametitle{Classificação de FSM}
    \includegraphics[height = 2.5in, width = 4in]{modelo_4.png}
\end{frame}

\begin{frame}
  \frametitle{Classificação de FSM}
    \includegraphics[height = 2in.5, width = 4in]{modelo_5.png}
\end{frame}

\begin{frame}
  \frametitle{Classificação de FSM}
    \includegraphics[height = 2.5in, width = 4in]{modelo_6.png}
\end{frame}

\begin{frame}
  \frametitle{Classificação de FSM}
  \begin{itemize}
    \item A saída em uma máquina de estados pode ou não depender da entrada, com isto podemos definir o dois modelos: Moore e o Mealy.\pause
    \item Tanto as máquinas Moore e Mealy a única diferença entre os dois esta na forma como a saída é gerada.
  \end{itemize}
\end{frame}

\begin{frame}
  \frametitle{Classificação de FSM}
    \includegraphics[height = 1.3in, width = 4in]{modelo_7.png}
\end{frame}

\begin{frame}
  \frametitle{Classificação de FSM}
    \includegraphics[height = 1.3in, width = 4in]{modelo_8.png}
\end{frame}

\begin{frame}
  \frametitle{Classificação de FSM}
  \textbf{Deve-se ficar bem claro que ambas as máquinas são idênticas diferenciando-se somente na forma como a saída é produzida.}
\end{frame}

\begin{frame}
  \frametitle{Classificação de FSM}
  \begin{itemize}
    \item Descrevem as operações de uma máquina de estados finita.\pause
    \item Cada estado existente é representado por um nó e este nó é rotulado por um nome ou um código.\pause
    \item Para toda transição de estado da FSM, cada nó é conectador por uma linha, que pode ser rotulada ou não. Os rótulos indicam duas características:\pause
    \begin{enumerate}
     \item Linhas rotuladas indicam \textbf{condições} para uma transição.\pause
     \item Linhas sem rótulos indicam transições \textbf{incondicionais}. Neste caso \textit{somente} uma linha incondicional pode ser originada por estado.\pause
    \end{enumerate}
    \item Por uma questão de clareza entre os modelos adota-se a seguinte diferenciação no diagrama de estados:\pause
    \begin{enumerate}
     \item Moore: A saída é colocada dentro do estado.\pause
     \item Mealy: A saída é colocada sobre as linhas que indicam a transição.
    \end{enumerate}
  \end{itemize}
\end{frame}

\begin{frame}
  \frametitle{Diagrama de estados}
    \includegraphics[height = 1.7in, width = 3.5in]{mealyvsmoore.png}
\end{frame}

\begin{frame}
  \frametitle{Diagrama de estados}
    \includegraphics[height = 2.5in, width = 3in]{mealyvsmoore_2.png}
\end{frame}

\begin{frame}
  \frametitle{Diagrama de estados}
    \includegraphics[height = 2.5in, width = 3in]{mealyvsmoore_3.png}
\end{frame}

\begin{frame}
  \frametitle{Diagrama de estados}
    \includegraphics[height = 2.5in, width = 3in]{mealyvsmoore_4.png}
\end{frame}

\begin{frame}
  \frametitle{Diagrama de estados}
    \includegraphics[height = 2.5in, width = 3in]{mealyvsmoore_5.png}
\end{frame}

\begin{frame}
  \frametitle{Diagrama de estados}
    \includegraphics[height = 2.2in, width = 4in]{mealyvsmoore_6.png}
\end{frame}

\begin{frame}
 \frametitle{Detalhes das máquinas de estados}
  \begin{itemize}
   \item Lógica do estado seguinte\pause
   \begin{itemize}
    \item Circuito combinacional\pause
    \item Utilização apenas de portas lógicas (And, Or, Not, Xor, etc)\pause
    \item As equações geradas para secção devem \textbf{adequar-se as entradas dos elementos de memória, de modo que a saída deste mudem para o estado correto.} \pause
   \end{itemize}
    \item Elementos de memória\pause
      \begin{itemize}
       \item flip-flops que em suas saídas mostram o estado atual do circuito\pause
       \item cada flip-flops corresponde a uma variável de estado \pause
       \item n flip-flops permite a representação de estados \pause
       \item Usam-se circuitos MSI, como contadores ou registradores de deslocamento para simplificar os desenhos. 
      \end{itemize}
  \end{itemize}
\end{frame}

\begin{frame}
 \frametitle{Detalhes das máquinas de estados}
  \begin{itemize}
   \item Lógica de saída.\pause
   \begin{itemize}
    \item Circuito combinacional.\pause
    \item As saídas do circuito são geradas por intermédio de portas lógicas.\pause
    \item No caso Maley incluem-se as \textbf{entradas} do circuito além da entrada dessa seção que será a \textbf{saída dos elementos de memória.} \pause
    \item No caso Moore entrada dessa seção que será a \textbf{saída dos elementos de memória.}
   \end{itemize}
  \end{itemize}
\end{frame}

%SEÇÃO == ANALISE
\section{Análise}
\begin{frame}
 \frametitle{Análise}
  \begin{itemize}
   \item Assumindo a definição formal de máquina de Mealy:\pause
    \begin{itemize}
     \item Estado seguinte = F(estado actual, entrada)\pause
     \item Saída = G(estado actual, entrada)\pause
    \end{itemize}
   \item A primeira equação explicita que o estado seguinte estará determinado pela entrada presente e o estado presente (actual)  do circuito.\pause
   \item A segunda precisa que a saída da máquina é determinada pela mesmas duas  variáveis.\pause
   \item O objetivo da análise dos circuitos seqüênciais é \textbf{determinar o estado seguinte e as funções de saída}  para poder \textbf{prever o comportamento do circuito.}
  \end{itemize}
\end{frame}

\begin{frame}
 \frametitle{Análise}
 \begin{itemize}
  \item Determinar o estado seguinte e as funções de saídas F e G.\pause
  \item Usar F e G para construir uma tabela de estados/saídas que especifique por completo o estado seguinte e a saída do circuito para cada combinação possível de estado 
	actual e entrada.\pause
  \item Diagrama de estados que apresenta na forma gráfica a informação anterior (opcional). 
 \end{itemize}
\end{frame}

\begin{frame}
  \frametitle{Análise de circuitos sequenciais}
  Dado o circuito deseja-se obter a descrição precisa do mesmo, este processo é chamado de análise.
  \begin{block}{Etapas da análise}
   \begin{enumerate}
    \item Derive as \textbf{equações de excitação} do circuito de próximo estado (entrada dos FF).\pause
    \item Derive as \textbf{equações de próximo estado} substituindo as equações de excitação obtidas no passo anterior nas equações característica de cada FF.\pause
    \item Derive a \textbf{tabela do próximo estado} partindo das equações de próximo estado.\pause
    \item Derive as \textbf{equações de saída} do circuito.\pause
    \item Derive a tabela de saída para as equações de saída.\pause
    \item Faça o diagrama de estados utilizando a tabela de próximo estado e a tabela de saída.
   \end{enumerate}
  \end{block}
\end{frame}

%SUBSEÇÃO == EXEMPLO
\section{Exemplo}
\begin{frame}
  \frametitle{Exemplo}
  Aplicamos e detalhamos estes passos no exemplo a seguir.\\
  
  \textbf{Exemplo 1: Dado o circuito abaixo analise-o.}
   \includegraphics[height = 2.2in, width = 4in]{EXEMPLO_ANALISE_1.png}
\end{frame}

\begin{frame}
  \frametitle{Exemplo}
  \textbf{1 – Derive as equações de excitação}
  \begin{block}{\textbf{Equações de excitação}}
   As equações de excitação são extraídas do circuito combinacional, onde as saídas destes correspondem as entradas da memória dos estados em uma FSM. 
   Portanto temos uma equação para cada FF.
  \end{block}
\end{frame}

\begin{frame}
  \frametitle{Exemplo}
  \textbf{1 – Derive as equações de excitação}
   \includegraphics[height = 2.2in, width = 4in]{EXEMPLO_ANALISE_2.png}
\end{frame}

\begin{frame}
  \frametitle{Exemplo}
  \textbf{1 – Derive as equações de excitação}
  \begin{block}{\textbf{Equações de excitação do circuito}}
    \begin{enumerate}
     \item $ D_1 = AQ_1 + B\overline{Q_1}Q_0 $
     \item $ D_0 = (\overline{A}\odot\overline{Q_0}) = \overline{A}\overline{Q_0} + AQ_0 $
    \end{enumerate}
  \end{block}\pause
  Nada de mais, certo?
\end{frame}

\begin{frame}
  \frametitle{Exemplo}
  \textbf{2 – Derive as equações do próximo estado}
   \includegraphics[height = 2.2in, width = 4in]{EXEMPLO_ANALISE_3.png}
\end{frame}

\begin{frame}
  \frametitle{Exemplo}
  \textbf{2 – Derive as equações do próximo estado}
  \begin{block}{\textbf{Equações do próximo estado}}
    De acordo com as entradas dos FFs e com o comportamento específico de cada FF (este é dado pela equação característica) a \textbf{equação do próximo estado} 
    especificará qual o próximo estado da FSM.
  \end{block}
\end{frame}

\begin{frame}
  \frametitle{Exemplo}
  \textbf{2 – Derive as equações do próximo estado}
  \begin{block}{\textbf{Equações do próximo estado do circuito}}
    A equação característica do Flip-Flop D é Q = D, logo temos
    \begin{enumerate}
     \item $ Q_{1next} = AQ_1 + B\overline{Q_1}Q_0 $
     \item $ Q_{0next} = (\overline{A}\odot\overline{Q_0}) = \overline{A}\overline{Q_0} + AQ_0 $
    \end{enumerate}
  \end{block}\pause
  Obs.: O Flip-flop D sem Enable é um caso especial, em geral ao se aplicar a equação característica a expressão $ Q \neq D $.
\end{frame}

\begin{frame}
  \frametitle{Exemplo}
  \textbf{3 – Tabela de próximo estado}
  \begin{block}{\textbf{Tabela de próximo estado}}
    A tabela do próximo estado é simplesmente uma tabela verdade derivada das equações do próximo estado. O valor destes próximos estados são obtidos simplesmente 
    substituindo o valor do estado atual e o valor da entrada na equação do próximo estado apropriada. 
  \end{block}
\end{frame}

\begin{frame}
  \frametitle{Exemplo}
  \textbf{3 – Tabela de próximo estado}

  \begin{center}
    \begin{tabular}{|c|c|}
      \hline
		      & Próximo estado\\
	Estado atual  & $Q_{1next}Q_{0next}$ \\
	 $Q_1Q_0$     & AB = \\
		  & \begin{tabular}{c|c|c|c} 00 & 01 & 10 & 11\\ \end{tabular} \\
      \hline
	00 \pause & \begin{tabular}{c|c|c|c} 01 \pause & 01 \pause & 00 \pause & 00 \pause  \\ \end{tabular} \\
      \hline
	01 \pause & \begin{tabular}{c|c|c|c} 00 \pause & 10 \pause & 01 \pause & 11 \pause  \\ \end{tabular} \\	
      \hline
	10 \pause & \begin{tabular}{c|c|c|c} 01 \pause & 01 \pause & 10 \pause & 10 \pause  \\ \end{tabular} \\
      \hline
	11 \pause & \begin{tabular}{c|c|c|c} 00 \pause & 00 \pause & 11 \pause & 11 \pause  \\ \end{tabular} \\
      \hline
  \end{tabular}
 \end{center}
\end{frame}

\begin{frame}
  \frametitle{Exemplo}
  \textbf{3 – Tabela de próximo estado}
  \begin{itemize}
   \item Estado atual: Coluna que contém todos os possíveis estados que a máquina pode atingir. Esta diretamente relacionada ao número de FF e em geral 
	 conseguimos preenchê-la logo de início.\pause
   \item Próximo estado: É a coluna que descreve qual será o próximo estado de acordo com as entradas. Repare que esta pode possuir n colunas onde, cada 
	 uma representa uma combinação das entradas.
  \end{itemize}
\end{frame}

\begin{frame}
  \frametitle{Exemplo}
  \textbf{4 – Equações de saída}
   \includegraphics[height = 2.2in, width = 4in]{EXEMPLO_ANALISE_4.png}
\end{frame}

\begin{frame}
  \frametitle{Exemplo}
  \textbf{4 – Equações de saída}
   \begin{block}{\textbf{Equações de saída}}
    As equações de saída são derivadas do circuito combinacional da saída da FSM.
   \end{block}
\end{frame}

\begin{frame}
  \frametitle{Exemplo}
  \textbf{4 – Equações de saída}
   \begin{block}{\textbf{Equações de saída do circuito}}
     \begin{itemize}
      \item $X = Q_1 + \overline{Q_0}$
      \item $Y = \overline{(\overline{Q_1}Q_0)} = Q_1 + \overline{Q_0} $ 
     \end{itemize}
   \end{block}
\end{frame}

\begin{frame}
  \frametitle{Exemplo}
  \textbf{5 – Tabela de saída}
   \begin{block}{\textbf{Tabela de saída}}
    Como a tabela de próximo estado, a tabela de saída é uma tabela verdade que é derivada das equações de saída. 
   \end{block}
\end{frame}

\begin{frame}
  \frametitle{Exemplo}
  \textbf{5 – Tabela de saída}
  \begin{center}
    \begin{tabular}{|c|c|}
      \hline
	Estado atual & Saída \\
	$Q_1Q_0$ & $Q_{1next}Q_{0next}$ \\
	  & \begin{tabular}{cc} X & Y \\ \end{tabular} \\
      \hline
	00 \pause & \begin{tabular}{c|c} 1 \pause & 0 \pause\\ \end{tabular} \\
      \hline
	01 \pause & \begin{tabular}{c|c} 0 \pause & 0 \pause\\ \end{tabular} \\
      \hline
	10 \pause & \begin{tabular}{c|c} 1 \pause & 1 \pause\\ \end{tabular} \\
      \hline
	11 \pause & \begin{tabular}{c|c} 1 \pause & 1 \pause\\ \end{tabular} \\
      \hline
  \end{tabular}
 \end{center}
\end{frame}

\begin{frame}
  \frametitle{Exemplo}
  \textbf{6 – Diagrama de estados} \\
   O diagrama de estados dispensa apresentações formais. Segue a criação do diagrama de estados para o circuito do exemplo.
\end{frame}

\begin{frame}
 \frametitle{Exemplo}
 \textbf{6 – Diagrama de estados}
  \begin{itemize}
   \item Cada linha da tabela indica um estado, portanto nós teremos quatro estados para analisar.\pause 
   \item Temos duas entradas (A e B), que nós gera quatro possibilidades diferentes de entradas em para cada estado.\pause
   \item Começamos do estado inicial.
  \end{itemize}
\end{frame}

\begin{frame}
  \frametitle{Exemplo}
  \textbf{6 – Diagrama de estados}
  \begin{center}
   \includegraphics[height = 1.5in, width = 3.5in]{Diagrama_estado_tabela.png}
  \end{center}

  \begin{itemize}
   \item Analisando a tabela é possível notar que no estado 00 com a entrada A = 1, sempre leva ao mesmo estado independentemente 
	 de B. Logo temos AB = 1x em 00 leva a ele mesmo.\pause 
   \item Repare que se A = 0 no estado 00, teremos como próximo estado 01. Logo temos AB = 0x leva ao estado 01.
  \end{itemize} 
\end{frame}

\begin{frame}
  \frametitle{Exemplo}
  \textbf{6 – Diagrama de estados}
   \begin{center}
    \includegraphics[height = 2.2in, width = 1.7in]{Diagrama_de_estado_ex1.png}
   \end{center}
\end{frame}

\begin{frame}
  \frametitle{Exemplo}
  \textbf{6 – Diagrama de estados}
   \begin{center}
    \includegraphics[height = 2.2in, width = 2.5in]{Diagrama_de_estado_ex2.png}
   \end{center}
\end{frame}

\begin{frame}
  \frametitle{Exemplo}
  \textbf{6 – Diagrama de estados}
   \begin{center}
    \includegraphics[height = 2.2in, width = 2.5in]{Diagrama_de_estado_ex3.png}
   \end{center}
\end{frame}

\begin{frame}
  \frametitle{Exemplo}
  \textbf{6 – Diagrama de estados}
  \begin{center}
   \includegraphics[height = 1in, width = 2.7in]{Diagrama_estado_tabela_2.png}
  \end{center}

  \begin{itemize}
   \item Agora analisamos a segunda linha da tabela correspondente ao estado 01. E logo de início notamos que as saídas X e Y neste estado são iguais a 0.\pause
   \item Se AB = 10 então o circuito permanecerá em 01. \pause
   \item Se AB = 11 então o circuito irá para o estado 11. \pause
   \item Se AB = 00 então o circuito irá para o estado 00. \pause
   \item Se AB = 01 então o circuito irá para o estado 10. 
  \end{itemize} 
\end{frame}

\begin{frame}
  \frametitle{Exemplo}
  \textbf{6 – Diagrama de estados}
   \begin{center}
    \includegraphics[height = 2.2in, width = 2.5in]{Diagrama_de_estado_ex4.png}
   \end{center}
\end{frame}

\begin{frame}
  \frametitle{Exemplo}
  \textbf{6 – Diagrama de estados}
  \begin{center}
   \includegraphics[height = 1in, width = 2.7in]{Diagrama_estado_tabela_3.png}
  \end{center}

  \begin{itemize}
   \item O resto da montagem do diagrama segue o mesmo procedimento. 
  \end{itemize} 
\end{frame}

\begin{frame}
  \frametitle{Exemplo}
  \textbf{6 – Diagrama de estados}
  \begin{center}
   \includegraphics[height = 2.2in, width = 2.5in]{Diagrama_de_estado_ex5.png}
  \end{center}
\end{frame}

\begin{frame}
  \frametitle{Exemplo}
  \textbf{6 – Diagrama de estados}
  \begin{center}
   \includegraphics[height = 1in, width = 3in]{Diagrama_estado_tabela_4.png}
  \end{center}
\end{frame}

\begin{frame}
  \frametitle{Exemplo}
  \textbf{6 – Diagrama de estados}
  \begin{center}
   \includegraphics[height = 2.2in, width = 2.5in]{Diagrama_de_estado_ex6.png}
  \end{center}
\end{frame}

\begin{frame}
  \frametitle{Exemplo}
  \textbf{6 – Diagrama de estados}
  \begin{center}
   \includegraphics[height = 2.2in, width = 2.5in]{Diagrama_de_estado_ex7.png}
  \end{center}
\end{frame}


%SEÇÃO == BIBLIOGRAFIA
\section{Bibliografia}
\begin{frame}
 \frametitle{Referências bibliográficas}
 \begin{enumerate}
  \item Fundamentos Matemáticos para a Ciência da computação - Judith L Gersting
  \item Ewang, E. O., Digital Logic and Microprocessor Design with VHDL, 3o edição, Prentice Hall, 1999.
  \item Wakerly, J. F., Digital Designs Principles and Practices, 3o edição, Prentice Hall, 1999.
 \end{enumerate}
\end{frame}

%SEÇÃO == COLABORADORES
\section{Colaboradores}
\begin{frame}
 \frametitle{Colaboradores}
 \begin{enumerate}
  \item Rodrigo Siqueira de Melo
 \end{enumerate}
\end{frame}


\end{document}

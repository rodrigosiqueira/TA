\documentclass[10pt, compress, aspectratio=169]{beamer}

\usetheme[numbering=fraction, progressbar=none, titleformat=smallcaps]{metropolis}
\usepackage{booktabs}
\usepackage{array}
\usepackage{listings}
\usepackage{graphicx}
\usepackage[scale=2]{ccicons}
\usepackage{url}
\usepackage{relsize}
\usepackage{wasysym}

\usepackage{pgfplots}
\usepgfplotslibrary{dateplot}

\lstset{ %
  backgroundcolor={},
  basicstyle=\ttfamily\footnotesize,
  breakatwhitespace=true,
  breaklines=true,
  captionpos=n,
  commentstyle=\color{orange},
  escapeinside={\%*}{*)},
  extendedchars=true,
  frame=n,
  keywordstyle=\color{orange},
  language=bash,
  rulecolor=\color{black},
  showspaces=false,
  showstringspaces=false,
  showtabs=false,
  numbers=left,
  numbersep=3pt,
  stepnumber=1,
  stringstyle=\color{gray},
  tabsize=2,
  keywords={thrust,plus,device_vector, copy,transform,begin,end, copyin,
  copyout, acc, \_\_global\_\_, void, int, float, main, threadIdx, blockIdx,
  blockDim, if, else, malloc, NULL, cudaMalloc, cudaMemcpy, cudaSuccess,
  cudaGetLastError, cudaDeviceSynchronize, cudaFree, cudaMemcpyDeviceToHost,
  cudaMemcpyHostToDevice, const, data, independent, kernels, loop,
  fprintf, stderr, cudaGetErrorString, EXIT_FAILURE, for, dim3},
  otherkeywords={::, \#pragma, \#include, <<<,>>>, \&, \*, +, -, /, [, ], >, <}
}

\renewcommand*{\UrlFont}{\ttfamily\smaller\relax}

\graphicspath{{images/}}

\title{O projeto Debian precisa de você!}
\author{\footnotesize Rodrigo Siqueira \\ {\scriptsize rodrigosiqueiramelo@gmail.com} \\ {\scriptsize \url{http://siqueira.tech}} }
\institute{\includegraphics[height=2cm]{debian}\\[0.2cm]}

\begin{document}

\maketitle

%------------------------------------------------------------------------------
\section{Introdução}
\begin{frame}{Introdução}
  \begin{figure}
    \centering
    \includegraphics[width=\linewidth,
                     height=0.8\textheight,
                     keepaspectratio]{want_you}
    \caption{O projeto Debian quer você!}
  \end{figure}
\end{frame}

%------------------------------------------------------------------------------
\section{Utilizando o Debian}

\begin{frame}{Utilizando o Debian}

  \metroset{block=fill}
  \begin{exampleblock}{Utilizar o Debian já é uma contribuição! Ao instalar o Debian, você:}
      \begin{itemize}
        \item Estará testando o instalador
        \item Ficará familiarizado com sistemas GNU/Linux
      \end{itemize}
  \end{exampleblock}

  Existem boas opções para potencializar o uso do Debian como contribuição
  para a comunidade:

  \begin{itemize}
    \item Habilitar a submissão do \textit{popularity-contest} para os pacotes
          que você utiliza, isto permite que a comunidade saiba quais pacotes
          são mais úteis e populares
    \item Fazer upload de \textit{screenshots} de pacotes que você utiliza para
          que outros usuários possam conhecer como um determinado software se
          parece antes de utilizar ele
  \end{itemize}

  \metroset{block=fill}
  \begin{exampleblock}{Links importantes}
    \begin{itemize}
      \item \url{https://www.debian.org/CD}
      \item \url{http://get.debian.net}
    \end{itemize}
  \end{exampleblock}

\end{frame}

%------------------------------------------------------------------------------
\section{Eventos}

\begin{frame}{Eventos}
  \begin{itemize}
    \item Você pode ajudar com \textit{face pública} do Debian contribuindo com
          o site
    \item Auxiliar com a organização de eventos em todo o mundo
    \item Ajudar a promover o Debian falando e demonstrando a distribuição
    \item Você pode ajudar com a conferência anual do Debian! Existem várias
          tarefas, tais quais: ajudar na gravação de vídeos, receber as pessoas
          que chegam ao evento, auxiliar palestrantes, dentre outras tarefas.
    \item Outro eventos: mini-DebConfs, Debian Day Parties, Release Parties,
          Sprints de desenvolvimento, dentre outros.
  \end{itemize}

  \metroset{block=fill}
  \begin{exampleblock}{Links importantes}
    \begin{itemize}
      \item Debian Press Release: \url{https://www.debian.org/News}
      \item Debian Project News: \url{https://www.debian.org/News/weekly}
      \item Debian Bits, the Debian Blog: \url{https://bits.debian.org}
      \item Notícias em português: \url{http://debianbrasil.org.br}
    \end{itemize}
  \end{exampleblock}

\end{frame}

%------------------------------------------------------------------------------
\section{Suporte ao Usuário}

\begin{frame}{Suporte ao Usuário}
  \begin{itemize}
    \item Você já tem uma certa experiência com Debian? O que acha de ajudar
          respondendo perguntas na lista de email? Esta sem dúvida é uma ótima
          forma de contribuir
    \item Ajudar o usuário no IRC também ajuda muito
  \end{itemize}
\end{frame}

%------------------------------------------------------------------------------
\section{Testes e Bug Report}

\begin{frame}{Testes e Bug Report}
  \begin{itemize}
    \item Testar o Debian e os seus programas. Se encontrar algum erro, reporte
          o problema no Debian por meio de um relatório de erro para que os
          desenvolvedores/mantenedores fiquem cientes do problema
    \item Olhe os bugs abertos e tente reproduzir e descreva a sua experiência
          ao tentar replicar
  \end{itemize}

  \metroset{block=fill}
  \begin{exampleblock}{Links importantes}
    \begin{itemize}
      \item Como enviar relatórios de erro:  \url{https://www.debian.org/Bugs/Reporting}
      \item 7 dicas para enviar relatório de bug: \url{https://raphaelhertzog.com/go/bugreporting}
    \end{itemize}
  \end{exampleblock}

\end{frame}

%------------------------------------------------------------------------------
\section{Tradução}

\begin{frame}{Tradução}
  Você pode auxiliar com a tradução de aplicações ou informações relacionadas
  ao Debian

  \metroset{block=fill}
  \begin{exampleblock}{Links importantes}
    \begin{itemize}
      \item Página do time brasileiro: \url{https://wiki.debian.org/Brasil/Traduzir}
    \end{itemize}
  \end{exampleblock}

\end{frame}

%------------------------------------------------------------------------------
\section{Documentação}

\begin{frame}{Documentação}
  \begin{itemize}
    \item Você pode ajudar contribuindo com a documentação oficial ou com
          a Wiki do Debian. Alguns exemplos:
      \begin{itemize}
        \item Manuais
        \item Tutoriais
        \item HOWTOs
        \item FAQs
      \end{itemize}
    \item Você pode categorizar pacotes no site debtags para que os usuários
          possam encontrar o software que procuram com mais facilidade
  \end{itemize}

  \metroset{block=fill}
  \begin{exampleblock}{Links importantes}
    \begin{itemize}
      \item \url{https://www.debian.org/doc}
      \item \url{https://wiki.debian.org/Brasil/Documentos}
    \end{itemize}
  \end{exampleblock}
\end{frame}

%------------------------------------------------------------------------------
\section{Programação}

\begin{frame}{Programação}
  \begin{itemize}
    \item Você pode ajudar rastreando, encontrando e consertando falhas de
          segurança
    \item Você também pode ajudar a melhorar pacotes, repositórios, imagens e
          outras coisas
    \item Você pode contribuir com aplicações que você já utiliza mandando
          patches
    \item Você pode ajudar a portar o Debian para outras arquiteturas
    \item Você pode auxiliar com os software utilizados pelo Debian
  \end{itemize}
\end{frame}

%------------------------------------------------------------------------------
\section{Jurídico}

\begin{frame}{Jurídico}
  Advogado(a)s podem ajudar muita a comunidade com questões legais
\end{frame}

%------------------------------------------------------------------------------
\section{Empacotamento}

\begin{frame}{Empacotamento}
  \begin{itemize}
    \item Pode ajudar a manter pacotes fazendo parte de algum time
    \item Você pode mandar patches com melhorias ou correções para mantenedores
          de pacotes
    \item Você pode adotar pacotes
    \item Você pode empacotar aplicações novas
  \end{itemize}
\end{frame}

%------------------------------------------------------------------------------
\section{Doação}

\begin{frame}{Doação}
  Você pode doar equipamentos e serviços para o Debian
\end{frame}

%------------------------------------------------------------------------------
\section{About this presentation}
\begin{frame}[standout]
    Este trabalho é derivado da apresentação "O projeto Debian quer você!" de
    Paulo Henrique de Lima Santana
   \begin{center}\ccbysa\end{center}
\end{frame}

\maketitle

\end{document}

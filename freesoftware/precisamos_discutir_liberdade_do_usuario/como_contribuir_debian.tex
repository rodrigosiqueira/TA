\documentclass[10pt, compress, aspectratio=169]{beamer}

\usetheme[numbering=fraction, progressbar=none, titleformat=smallcaps]{metropolis}
\usepackage{booktabs}
\usepackage{array}
\usepackage{listings}
\usepackage{graphicx}
\usepackage[scale=2]{ccicons}
\usepackage{url}
\usepackage{relsize}
\usepackage{wasysym}

\usepackage{pgfplots}
\usepgfplotslibrary{dateplot}

\lstset{ %
  backgroundcolor={},
  basicstyle=\ttfamily\footnotesize,
  breakatwhitespace=true,
  breaklines=true,
  captionpos=n,
  commentstyle=\color{orange},
  escapeinside={\%*}{*)},
  extendedchars=true,
  frame=n,
  keywordstyle=\color{orange},
  language=bash,
  rulecolor=\color{black},
  showspaces=false,
  showstringspaces=false,
  showtabs=false,
  numbers=left,
  numbersep=3pt,
  stepnumber=1,
  stringstyle=\color{gray},
  tabsize=2,
  keywords={thrust,plus,device_vector, copy,transform,begin,end, copyin,
  copyout, acc, \_\_global\_\_, void, int, float, main, threadIdx, blockIdx,
  blockDim, if, else, malloc, NULL, cudaMalloc, cudaMemcpy, cudaSuccess,
  cudaGetLastError, cudaDeviceSynchronize, cudaFree, cudaMemcpyDeviceToHost,
  cudaMemcpyHostToDevice, const, data, independent, kernels, loop,
  fprintf, stderr, cudaGetErrorString, EXIT_FAILURE, for, dim3},
  otherkeywords={::, \#pragma, \#include, <<<,>>>, \&, \*, +, -, /, [, ], >, <}
}

\renewcommand*{\UrlFont}{\ttfamily\smaller\relax}

\graphicspath{{images/}}

\title{O projeto Debian precisa de você!}
\author{\footnotesize Rodrigo Siqueira \\ {\scriptsize rodrigosiqueiramelo@gmail.com} \\ {\scriptsize \url{http://siqueira.tech}} }
\institute{\includegraphics[height=2cm]{debian}\\[0.2cm]}

\begin{document}

\maketitle

%------------------------------------------------------------------------------
\section{Introdução}
\begin{frame}{Introdução}
  \begin{itemize}
    \item A liberdade de executar o programa como você desejar, para qualquer
          propósito (liberdade 0).
    \item A liberdade de estudar como o programa funciona, e adaptá-lo às suas
          necessidades (liberdade 1). Para tanto, acesso ao código-fonte é um
          pré-requisito.
    \item A liberdade de redistribuir cópias de modo que você possa ajudar
          outros (liberdade 2).
    \item A liberdade de distribuir cópias de suas versões modificadas a outros
          (liberdade 3). Desta forma, você pode dar a toda comunidade a chance
          de beneficiar de suas mudanças. Para tanto, acesso ao código-fonte é
          um pré-requisito.

  \end{itemize}
\end{frame}

%------------------------------------------------------------------------------
\section{Utilizando o Debian}


%------------------------------------------------------------------------------
\section{Doação}

\begin{frame}{Doação}
  Você pode doar equipamentos e serviços para o Debian
\end{frame}

%------------------------------------------------------------------------------
\section{About this presentation}
\begin{frame}[standout]
    Este trabalho é derivado da apresentação "O projeto Debian quer você!" de
    Paulo Henrique de Lima Santana
   \begin{center}\ccbysa\end{center}
\end{frame}

\maketitle

\end{document}
